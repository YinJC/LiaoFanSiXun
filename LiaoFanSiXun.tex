%!TEX TS-program = xelatex
%! TEX encoding = UTF-8 Unicode

%========================================全文布局
\documentclass[12pt,twoside,openany]{book}
\usepackage[screen,paperheight=14.4cm,paperwidth=10.8cm,
left=2mm,right=2mm,top=2mm,bottom=5mm]{geometry}

\usepackage[]{microtype}
\usepackage{graphicx}
\usepackage{amssymb,amsmath}
\usepackage{booktabs}
\usepackage{titletoc}
\usepackage{titlesec}
\usepackage{tikz}
\usepackage{enumerate}
\usepackage{wallpaper}
\usepackage{indentfirst}
%========================================设置字体
\usepackage[CJKnumber]{xeCJK}
\usepackage{xpinyin}
\setCJKmainfont[BoldFont={Adobe Heiti Std R}]{Hiragino Sans GB W3}
\setCJKfamilyfont{kai}{Adobe Kaiti Std R}
\setCJKfamilyfont{hei}{Adobe Heiti Std R}
\setCJKfamilyfont{fsong}{Adobe Fangsong Std R}

\newcommand{\kai}[1]{{\CJKfamily{kai}#1}}
\newcommand{\hei}[1]{{\CJKfamily{hei}#1}}
\newcommand{\fsong}[1]{{\CJKfamily{fsong}#1}}

\renewcommand\contentsname{目~录~}
\renewcommand\listfigurename{图~列~表~}
\renewcommand\listtablename{表~目~录~}

%========================================章节样式
\titlecontents{chapter}
[0em]
{}
{\large\CJKfamily{hei}{}}
{}{\dotfill\contentspage}%用点填充
%
\titlecontents{section}
[4em]
{}
{\thecontentslabel\quad}
{}{\titlerule*{.}\contentspage}

\titleformat{\chapter}[display]
	{\CJKfamily{fsong}\Large\centering}
	{\titlerule[1pt]%
	 \filleft%
	}
	{-7ex}
	{\Huge
	 \filright}
	[{\titlerule[1pt]}]

%========================================设置目录
\usepackage[setpagesize=false,
            linkcolor=black,
            colorlinks, %注释掉此项则交叉引用为彩色边框(将colorlinks和pdfborder同时注释掉)
            pdfborder=001   %注释掉此项则交叉引用为彩色边框
            ]{hyperref}

\setlength{\parindent}{2em} %首行缩进
\linespread{1.2}              %行距
\setlength{\parskip}{15pt}    %段距

%========================================页眉页脚
\usepackage{fancyhdr}
\pagestyle{fancy}
\fancyhf{}
\fancyfoot{}
\fancyfoot[LE,RO]{\thepage}
\setlength{\footskip}{6pt}
%========================================标题作者
\title{了凡四训}
\author{袁了凡}
\date{}

%========================================正文
\begin{document}
\TileSquareWallPaper{1}{TGTamber}%背景图片

\maketitle
\tableofcontents
\newpage


%\clearpage\cleardoublepage \phantomsection
%\setcounter{chapter}{0}
%\chapter{第一章\ 网上偶遇}
\chapter*{序}\addcontentsline{toc}{chapter}{\large\CJKfamily{hei}序}
《了凡四训》这本书,是中国明朝袁了凡先生所作的家训,教戒他的儿子袁天启,认识命运的真相,明辨善恶的标准,改过迁善的方法,以及行善积德谦虚种种的效验;并且以他自己改造命运的经验来“现身说法”;读了可以使人心目豁开,信心勇气倍增,亟欲效法了凡先生,来改造自己的命运;实在是一本有益世道人心,转移社会风气不可多得的好书。

袁了凡(1533-1606),出生于嘉善县魏塘镇,初名表,后改名黄,字庆远,又字坤仪、仪甫,初号学海,后改了凡,后人常以其号了凡称之。袁了凡是明朝重要思想家,是迄今所知中国第一位具名的善书作者。他的《了凡四训》融会禅学与理学,劝人积善改过,强调从治心入手的自我修养,提倡记功过格,在社会上流行一时。

\part{原文}
\chapter{第一篇\ 立命之学}

余童年丧父,老母命弃举业学医,谓可以养生,可以济人,且习一艺以成名,尔父\xpinyin*{夙}心也。
后余在慈云寺,遇一老者,修\xpinyin*{髯}伟貌,飘飘若仙,余敬礼之。语余曰:「子仕路中人也,明年即进学,何不读书?」

余告以故,并叩老者姓氏里居。

曰:「吾姓孔,云南人也。得邵子皇极数正传,数该传汝。」

余引之归,告母。

母曰:「善待之。」

试其数,\xpinyin*{纤悉皆验}。余遂启读书之念,谋之表兄沈称,言:「郁海谷先生,在沈友夫家开馆,我送汝寄学甚便。」

余遂礼郁为师。

孔为余起数:县考童生,当十四名;府考七十一名,提学考第九名。明年赴考,三处名数皆合。
复为\xpinyin{卜}{bu3}终身休\xpinyin*{咎},言:某年考第几名,某年当补\xpinyin*{廪},某年当贡,贡后某年,当选四川一大尹,在任三年半,即宜告归。
五十三岁八月十四日丑时,当终于正寝,惜无子。余备录而谨记之。

自此以后,凡遇考校,其名数先后,皆不出孔公所悬定者。独算余食廪米九十一石五斗当出贡;及食米七十一石,屠宗师即批准补贡,余窃疑之。
后果为署印杨公所驳,直至丁卯年(公元1567年),\xpinyin*{殷秋溟}宗师见余场中备卷,叹曰:「五策,即五篇奏议也,岂可使\xpinyin*{博洽淹贯}之儒,老于窗下乎!」
遂依县申文准贡,连前食米计之,实九十一石五斗也。余因此益信进退有命,迟速有时,\xpinyin*{澹}然无求矣。

贡入燕都,留京一年,终日静坐,不阅文字。己巳(公元1569年)归,游南\xpinyin*{雍},未入监,先访云谷会禅师于\xpinyin*{栖}霞山中,对坐一室,凡三昼夜不瞑目。

云谷问曰:「凡人所以不得作圣者,只为妄念相缠耳。汝坐三日,不见起一妄念,何也?」

余曰:「吾为孔先生算定,荣辱生死,皆有定数,即要妄想,亦无可妄想。」

云谷笑曰:「我待汝是豪杰,原来只是凡夫。」

问其故?

曰:「人未能无心,终为阴阳所缚,安得无数?但惟凡人有数;极善之人,数固拘他不定;极恶之人,数亦拘他不定。汝二十年来,被他算定,不曾转动一毫,岂非是凡夫?」

余问曰:「然则数可逃乎?」

曰:「命由我作,福自己求。诗书所称,的为明训。我教典中说:『求富贵得富贵,求男女得男女,求长寿得长寿。』夫妄语乃释迦大戒,诸佛菩萨,岂\xpinyin*{诳}语欺人?」

余进曰:「孟子言:『求则得之』,是求在我者也。道德仁义可以力求;功名富贵,如何求得?」

云谷曰:「孟子之言不错,汝自错解耳。汝不见六祖说:『一切福田,不离方寸;从心而觅,感无不通。』求在我,不独得道德仁义,亦得功名富贵;内外双得,是求有益于得也。若不反躬内省,而徒向外驰求,则求之有道,而得之有命矣,内外双失,故无益。」

因问:「孔公算汝终身若何?」

余以实告。

云谷曰:「汝自\xpinyin*{揣}应得科第否?应生子否?」

余追\xpinyin{省}{xing3}良久,曰:「不应也。科第中人,类有福相,余福薄,又不能积功累行,以基厚福;兼不耐烦剧,不能容人;时或以才智盖人,直心直行,轻言妄谈。凡此皆薄福之相也,岂宜科第哉。

地之\xpinyin*{秽}者多生物,水之清者常无鱼;余好洁,宜无子者一;和气能育万物,余善怒,宜无子者二;爱为生生之本,忍为不育之根;余\xpinyin*{矜}惜名节,常不能舍己救人,宜无子者三; 多言耗气,宜无子者四;喜饮\xpinyin*{铄}精,宜无子者五; 好彻夜长坐,而不知\xpinyin*{葆元毓神},宜无子者六。其\xpinyin*{馀}过恶尚多,不能悉数。」

云谷曰:「岂惟科第哉。世间享千金之产者,定是千金人物;享百金之产者,定是百金人物;应饿死者,定是饿死人物;天不过因材而\xpinyin*{笃},几曾加\xpinyin*{纤}毫意思。

即如生子,有百世之德者,定有百世子孙保之;有十世之德者,定有十世子孙保之;有三世二世之德者,定有三世二世子孙保之;其斩焉无后者,德至薄也。

汝今既知非。将向来不发科第,及不生子之相,尽情改刷;务要积德,务要包荒,务要和爱,务要惜精神。从前种种,譬如昨日死;从后种种,譬如今日生;此义理再生之身。

夫血肉之身,尚然有数;义理之身,岂不能格天。太甲曰:『天作孽,犹可违;自作孽,不可活。』诗云:『永言配命,自求多福。』孔先生算汝不登科第,不生子者,此天作之孽,犹可得而违;汝今扩充德性,力行善事,多积阴德,此自己所作之福也,安得而不受享乎?

易为君子\xpinyin*{谋},趋吉避凶;若言天命有常,吉何可趋,凶何可避?开章第一义,便说:『积善之家,必有馀庆。』汝信得及否?」

余信其言,拜而受教。因将往日之罪,佛前尽情发露,为疏一通,先求登科;誓行善事三千条,以报天地祖宗之德。

云谷出功过格示余,令所行之事,逐日登记;善则记数,恶则退除,且教持准提咒,以期必验。

语余曰:「符\xpinyin*{箓}家有云:『不会书符,被鬼神笑。』此有秘传,只是不动念也。执笔书符,先把万缘放下,一尘不起。从此念头不动处,下一点,谓之\xpinyin*{混沌}开基。由此而一笔挥成,更无思虑,此符便灵。凡祈天立命,都要从无思无虑处感格。

孟子论立命之学,而曰:『夭寿不贰。』夫夭寿,至贰者也。当其不动念时,\xpinyin*{孰}为夭,孰为寿?细分之,丰歉不贰,然后可立贫富之命;穷通不贰,然后可立贵贱之命;夭寿不贰,然后可立生死之命。人生世间,惟死生为重,曰夭寿,则一切顺逆皆该之矣。

至修身以\xpinyin{俟}{si4}之,乃积德祈天之事。\xpinyin*{曰}修,则身有过恶,皆当治而去之;曰俟,则一毫\xpinyin*{觊觎},一毫将迎,皆当斩绝之矣。到此地位,直造先天之境,即此便是实学。

汝未能无心,但能持准提咒,无记无数,不令间断,持得纯熟,于持中不持,于不持中持。到得念头不动,则灵验矣。」

余初号学海,是日改号了凡;盖悟立命之说,而不欲落凡夫\xpinyin*{窠臼}也。从此而后,终日\xpinyin*{兢兢},便觉与前不同。前日只是悠悠放任,到此自有\xpinyin*{战兢惕厉}景象,在暗室屋漏中,常恐得罪天地鬼神;遇人\xpinyin*{憎}我毁我,自能\xpinyin*{恬}然容受。

到明年(公元1570年)礼部考科举,孔先生算该第三,忽考第一;其言不验,而秋\xpinyin*{闱}中式矣。然行义未纯,检身多误;或见善而行之不勇,或救人而心常自疑;或身勉为善,而口有过言;或醒时操持,而醉后放逸;以过折功,日常虚度。自\xpinyin*{己巳}岁(公元1569年)发愿,直至己卯岁(公元1579年),历十馀年,而三千善行始完。

时方从李渐庵入关,未及回向。庚辰(公元1580年)南还。始请性空,慧空诸上人,就东塔禅堂回向。遂起求子愿,亦许行三千善事。辛巳(公元1581年),生男天启。

余行一事,随以笔记;汝母不能书,每行一事,\xpinyin*{辄}用鹅毛管,印一朱圈于历 日之上。或施食贫人,或买放生命,一日有多至十馀者。至癸未(公元1583年)八月,三千之数已满。复请性空辈,就家庭回向。九月十三日,复起求中进士愿, 许行善事一万条,丙\xpinyin*{戌}(公元1586年)登第,授宝\xpinyin{坻}{di3}知县。

余置空格一册,名曰治心篇。晨起坐堂,家人携付门役,置案上,所行善恶, 纤悉必记。夜则设桌于庭,效赵阅道焚香告帝。

汝母见所行不多,\xpinyin*{辄颦蹙曰}:「我前在家,相助为善,故三千之数得完;今许一万,衙中无事可行,何时得圆满乎?」

夜间偶梦见一神人,余言善事难完之故。神曰:「只减粮一节,万行俱完矣。」盖宝坻之田,每亩二分三厘七毫。余为区处,减至一分四厘六毫,委有此事,心颇惊疑。适幻余禅师自五台来,余以梦告之,且问此事宜信否?

师曰:「善心真切,即一行可当万善,况合县减粮,万民受福乎?」

吾即捐\xpinyin*{俸}银,请其就五台山斋僧一万而回向之。

孔公算予五十三岁有厄,余未尝祈寿,是岁竟无恙,今六十九矣。书曰:「天难\xpinyin*{谌},命\xpinyin*{靡}常。」又云:「惟命不于常」,皆非诳语。吾于是而知,凡称祸福自己求之者,乃圣贤之言。若谓祸福惟天所命,则世俗之论矣。

汝之命,未知若何?即命当荣显,常作落寞想;即时当顺利,常作\xpinyin*{拂逆}想;即眼前足食,常作\xpinyin*{贫窭}想;即人相爱敬,常作恐惧想;即家世望重,常作卑下想;即学问颇优,常作浅陋想。

远思扬祖宗之德,近思盖父母之\xpinyin*{愆};上思报国之恩,下思造家之福;外思济人之急,内思闲己之邪。

务要日日知非,日日改过;一日不知非,即一日安于自是; 一日无过可改,即一日无步可进;天下聪明俊秀不少,所以德不加修,业不加广者,只为因循二字,耽阁一生。

云谷禅师所授立命之说,乃至精至\xpinyin*{邃},至真至正之理,其熟玩而勉行之,\xpinyin*{毋}自旷也。

\chapter{第二篇\ 改过之法}
春秋诸大夫,见人言动,亿而谈其祸福,\xpinyin*{靡}不验者,左国诸记可观也。大都吉凶之兆,萌乎心而动乎四体,其过于厚者常获福,过于薄者常近祸,俗眼多\xpinyin*{翳},谓有未定而不可测者。至诚合天,福之将至,观其善而必先知之矣。祸之将至,观其不善而必先知之矣。今欲获福而远祸,未论行善,先须改过。

但改过者,第一,要发耻心。思古之圣贤,与我同为丈夫,彼何以百世可师?我何以一身瓦裂?\xpinyin*{耽}染尘情,私行不义,谓人不知,傲然无愧,将日沦于禽兽而不自知矣;世之可羞可耻者,莫大乎此。孟子曰:耻之于人大矣。以其得之则圣贤,失之则禽兽耳。此改过之要机也。

第二,要发畏心。天地在上,鬼神难欺,吾虽过在隐微,而天地鬼神,实鉴临之,重则降之百殃,轻则损其现福,吾何可以不惧?不惟此也。闲居之地,指视昭然;吾虽掩之甚密,文之甚巧,而肺肝早露,终难自欺;被人\xpinyin*{觑}破,不值一文矣,乌得不\xpinyin*{懔懔}?不惟是也。一息尚存,弥天之恶,犹可悔改;古人有一生作恶,临死悔悟,发一善念,遂得善终者。谓一念猛厉,足以\xpinyin*{涤}百年之恶也。譬如千年幽谷,一灯才照,则千年之暗俱除;故过不论久近,惟以改为贵。但尘世无常,肉身易\xpinyin*{殒},一息不属,欲改无由矣。明则千百年担负恶名,虽孝子慈孙,不能洗涤;幽则千百劫沉沦狱报,虽圣贤佛菩萨,不能援引。乌得不畏?

第三,须发勇心。人不改过,多是因循退缩;吾须奋然振作,不用迟疑,不烦等待。小者如芒刺在肉,速与\xpinyin*{抉剔};大者如毒蛇\xpinyin*{啮}指,速与斩除,无丝毫凝滞,此风雷之所以为益也。

具是三心,则有过斯改,如春冰遇日,何患不消乎?然人之过,有从事上改者,有从理上改者,有从心上改者;工夫不同,效验亦异。

如前日杀生,今戒不杀;前日怒\xpinyin*{詈},今戒不怒;此就其事而改之者也。强制于外,其难百倍,且病根终在,东灭西生,非究竟\xpinyin*{廓}然之道也。

善改过者,未禁其事,先明其理;如过在杀生,即思曰:上帝好生,物皆恋命,杀彼养己,岂能自安?且彼之杀也,既受屠割,复入鼎\xpinyin*{镬},种种痛苦,彻入骨髓;己之养也,珍膏罗列,食过即空,疏食菜\xpinyin*{羹},尽可充腹,何必\xpinyin*{戕}彼之生,损己之福哉?又思血气之属,皆含灵知,既有灵知,皆我一体;纵不能躬修至德,使之尊我亲我,岂可日戕物命,使之仇我憾我于无穷也?一思及此,将有对食痛心,不能下咽者矣。

如前日好怒,必思曰:人有不及,情所宜\xpinyin*{矜};\xpinyin*{悖}理相干,于我何与?本无可怒者。又思天下无自是之豪杰,亦无尤人之学问;有不得,皆己之德未修,感未至也。吾悉以自反,则\xpinyin*{谤}毁之来,皆磨炼玉成之地;我将欢然受赐,何怒之有?

又闻而不怒,虽\xpinyin*{谗焰薰天},如举火焚空,终将自息;闻谤而怒,虽巧心力辩,如春蚕作茧,自取缠绵;怒不惟无益,且有害也。其馀种种过恶,皆当据理思之。

此理既明,过将自止。

何谓从心而改?过有千端,惟心所造;吾心不动,过安从生?学者于好色,好名,好货,好怒,种种诸过,不必逐类寻求;但当一心为善,正念现前,邪念自然污染不上。如太阳当空,\xpinyin*{魍魉}潜消,此精一之真传也。过由心造,亦由心改,如斩毒树,直断其根,\xpinyin*{奚}必枝枝而伐,叶叶而摘哉?

大\xpinyin*{抵}最上治心,当下清净;才动即觉,觉之即无;苟未能然,须明理以遣之;又未能然,须随事以禁之;以上事而兼行下功,未为失策。执下而昧上,则拙矣。

顾发愿改过,明须良朋提醒,幽须鬼神证明;一心忏悔,昼夜不懈,经一七,二七,以至一月,二月,三月,必有效验。

或觉心神\xpinyin*{恬}旷;或觉智慧顿开;或处\xpinyin*{冗}\xpinyin{沓}{ta4}而触念皆通;或遇怨仇而回\xpinyin*{嗔}作喜;或梦吐黑物;或梦往圣先贤,提携接引;或梦飞步太虚;或梦\xpinyin*{幢幡}宝盖,种种胜事,皆过消罪灭之象也。然不得执此自高,画而不进。

昔\xpinyin*{蘧}伯玉当二十岁时,已觉前日之非而尽改之矣。至二十一岁,乃知前之所改,未尽也;及二十二岁,回视二十一岁,犹在梦中,岁复一岁,递递改之,行年五十,而犹知四十九年之非,古人改过之学如此。

吾辈身为凡流,过恶猬集,而回思往事,常若不见其有过者,心粗而眼翳也。 然人之过恶深重者,亦有效验:或心神昏塞,转头即忘;或无事而常烦恼;或见君子而\xpinyin*{赧}然相\xpinyin*{沮};或闻正论而不乐;或施惠而人反怨;或夜梦颠倒,甚则妄言失志;皆作孽之相也,苟一类此,即须奋发,舍旧图新,幸勿自误。

\chapter{第三篇\ 积善之方}
易曰:「积善之家,必有馀庆。」昔颜氏将以女妻叔梁\xpinyin{纥}{he2},而历叙其祖宗积德之长,逆知其子孙必有兴者。孔子称舜之大孝,曰:「宗庙\xpinyin*{飨}之,子孙保之」,皆至论也。试以往事\xpinyin*{徵}之。

杨少师荣,建宁人。世以济渡为生,久雨溪涨,横流冲毁民居,\xpinyin*{溺}死者顺流而下,他舟皆捞取货物,独少师曾祖及祖,惟救人,而货物一无所取,乡人\xpinyin*{嗤}其愚。\xpinyin*{逮}少师父生,家渐裕,有神人化为道者,语之曰:「汝祖父有阴功,子孙当贵显,宜葬某地。」遂依其所指而\xpinyin*{窆}之,即今白兔坟也。后生少师,弱冠登第,位至三公,加曾祖,祖,父,如其官。子孙贵盛,至今尚多贤者。

\xpinyin*{鄞}人杨自惩,初为县吏,存心仁厚,守法公平。时县宰严肃,偶\xpinyin*{挞}一囚,血流满前,而怒犹未息,杨跪而宽解之。宰曰:「怎奈此人越法悖理,不由人不怒。」

自惩叩首曰:「上失其道,民散久矣,如得其情,哀矜勿喜;喜且不可,而况怒乎?」宰为之\xpinyin*{霁}颜。

家甚贫,\xpinyin*{馈}遗一无所取,遇囚人乏粮,常多方以济之。一日,有新囚数人待\xpinyin*{哺},家又缺米;给囚则家人无食;自顾则囚人堪悯;与其妇商之。

妇曰:「囚从何来?」

曰:「自杭而来。沿路忍饥,菜色可\xpinyin*{掬}。」

因撤己之米,煮粥以食囚。后生二子,长曰守陈,次曰守址,为南北吏部侍郎;长孙为刑部侍郎;次孙为四川廉宪,又俱为名臣;今楚亭,德政,亦其裔也。

昔正统间,邓茂七倡乱于福建,士民从贼者甚众;朝廷起鄞县张都宪楷南征,以计擒贼,后委布政司谢都事,搜杀东路贼党;谢求贼中党附册籍,凡不附贼者,密授以白布小旗,约兵至日,插旗门首,戒军兵无妄杀,全活万人;后谢之子迁,中状元,为宰辅;孙丕,复中探花。

\xpinyin*{莆}田林氏,先世有老母好善,常作粉团施人,求取即与之,无倦色;一仙化为道人,每旦索食六七团。母日日与之,终三年如一日,乃知其诚也。因谓之曰:「吾食汝三年粉团,何以报汝?府后有一地,葬之,子孙官爵,有一升麻子之数。」

其子依所点葬之,初世即有九人登第,累代\xpinyin*{簪缨}甚盛,福建有无林不开榜之谣。

冯\xpinyin*{琢庵}太史之父,为\xpinyin*{邑庠}生。隆冬早起赴学,路遇一人,倒卧雪中,\xpinyin*{扪}之,半僵矣。遂解己绵裘衣之,且扶归救苏。梦神告之曰:「汝救人一命,出至诚心,吾遣韩琦为汝子。」及生琢庵,遂名琦。

台州应尚书,壮年习业于山中。夜鬼啸集,往往惊人,公不惧也;一夕闻鬼云:「某妇以夫久客不归,翁姑逼其嫁人。明夜当\xpinyin*{缢}死于此,吾得代矣。」公潜卖田,得银四两。即伪作其夫之书,寄银还家;其父母见书,以手迹不类,疑之。

既而曰:「书可假,银不可假,想儿无恙。」妇遂不嫁。其子后归,夫妇相保如初。

公又闻鬼语曰:「我当得代,奈此秀才坏吾事。」

旁一鬼曰:「尔何不祸之?」

曰:「上帝以此人心好,命作阴德尚书矣,吾何得而祸之?」

应公因此益自努励,善日加修,德日加厚;遇岁饥,辄捐谷以\xpinyin*{赈}之;遇亲戚有急,辄委曲维持;遇有横逆,辄反躬自责,怡然顺受;子孙登科第者,今累累也。

常熟徐凤竹\xpinyin*{栻},其父素富,偶遇年荒,先捐租以为同邑之倡,又分谷以赈贫乏,夜闻鬼唱于门曰:「千不\xpinyin*{诓},万不诓;徐家秀才,做到了举人郎。」相续而呼,连夜不断。是岁,凤竹果举于乡,其父因而益积德,\xpinyin*{孳孳}不怠,修桥修路,斋僧接众,凡有利益,无不尽心。后又闻鬼唱于门曰:「千不诓,万不诓;徐家举人,直做到都堂。」凤竹官终两浙巡抚。

嘉兴屠康僖公,初为刑部主事,宿狱中,细询诸囚情状,得无辜者若干人,公不自以为功,密疏其事,以白堂官。后朝审,堂官摘其语,以讯诸囚,无不服者,释冤抑十馀人。一时\xpinyin*{辇}下\xpinyin*{咸}颂尚书之明。

公复\xpinyin*{禀}曰:「\xpinyin*{辇毂}之下,尚多冤民,四海之广,兆民之众,岂无枉者?宜五年差一减刑官,核实而平反之。」

尚书为奏,允其议。时公亦差减刑之列,梦一神告之曰:「汝命无子,今减刑之议,深合天心,上帝赐汝三子,皆衣紫腰金。」是夕夫人有\xpinyin*{娠},后生应\xpinyin*{埙},应坤,应\xpinyin*{埈},皆显官。

嘉兴包凭,字信之,其父为池阳太守,生七子,凭最少,\xpinyin*{赘}平湖袁氏,与吾父往来甚厚,博学高才,累举不第,留心二氏之学。一日东游\xpinyin*{泖}湖,偶至一村寺中,见观音像,淋漓露立,即解\xpinyin*{橐}中十金,授主僧,令修屋宇,僧告以功大银少,不能竣事;复取松布四\xpinyin*{疋},检\xpinyin*{箧}中衣七件与之,内\xpinyin*{纻褶},系新置,其仆请已之。

凭曰:「但得圣像无恙,吾虽裸\xpinyin*{裎}何伤?」

僧垂泪曰:「舍银及衣布,犹非难事。只此一点心,如何易得。」

后功完,拉老父同游,宿寺中。公梦\xpinyin{伽}{qie2}蓝来谢曰:「汝子当享世禄矣。」后子\xpinyin*{汴},孙\xpinyin*{柽}芳,皆登第,作显官。

嘉善支立之父,为刑房吏,有囚无辜陷重辟,意哀之,欲求其生。囚语其妻曰:「支公嘉意,愧无以报,明日延之下乡,汝以身事之,彼或肯用意,则我可生也。」其妻泣而听命。及至,妻自出劝酒,具告以夫意。支不听,卒为尽力平反之。囚出狱,夫妻登门叩谢曰:「公如此厚德,晚世所稀,今无子,吾有弱女,送为\xpinyin*{箕帚}妾,此则礼之可通者。」支为备礼而纳之,生立,弱冠中魁,官至翰林孔目,立生高,高生禄,皆贡为学博。禄生大\xpinyin*{纶},登第。

凡此十条,所行不同,同归于善而已。若复精而言之,则善有真,有假;有端,有曲;有阴,有阳;有是,有非;有偏,有正;有半,有满;有大,有小;有难,有易;皆当深辨。为善而不穷理,则自谓行持,岂知造孽,枉费苦心,无益也。

何谓真假?昔有儒生数辈,\xpinyin*{谒}中峰和尚,问曰:「佛氏论善恶报应,如影随形。今某人善,而子孙不兴;某人恶,而家门隆盛;佛说无稽矣。」

中峰云:「凡情未\xpinyin*{涤},正眼未开,认善为恶,指恶为善,往往有之。不憾己之是非颠倒,而反怨天之报应有差乎?」

众曰:「善恶何致相反?」

中峰令试言。

一人谓「\xpinyin*{詈}人殴人是恶;敬人礼人是善。」

中峰云:「未必然也。」

一人谓「贪财妄取是恶,廉洁有守是善。」

中峰云:「未必然也。」

众人历言其状,中峰皆谓不然。因请问。

中峰告之曰:「有益于人,是善;有益于己,是恶。有益于人,则殴人,詈人皆善也;有益于己,则敬人,礼人皆恶也。是故人之行善,利人者公,公则为真;利己者私,私则为假。又根心者真,袭迹者假;又无为而为者真,有为而为者假;皆当自考。」

何谓端曲?今人见\xpinyin*{谨}愿之士,类称为善而取之;圣人则宁取狂狷。至于谨愿之士,虽一乡皆好,而必以为德之贼;是世人之善恶,分明与圣人相反。推此一端,种种取舍,无有不\xpinyin*{谬};天地鬼神之福善祸淫,皆与圣人同是非,而不与世俗同取舍。凡欲积善,决不可徇耳目,惟从心源隐微处,默默洗涤,纯是济世之心,则为端;苟有一毫媚世之心,即为曲;纯是爱人之心,则为端;有一毫愤世之心,即为曲;纯是敬人之心,则为端;有一毫玩世之心,即为曲;皆当细辨。

何谓阴阳?凡为善而人知之,则为阳善;为善而人不知,则为阴德。阴德,天报之;阳善,享世名。名,亦福也。名者,造物所忌;世之享盛名而实不副者,多有奇祸;人之无过\xpinyin*{咎}而横被恶名者,子孙往往骤发,阴阳之际微矣哉。

何谓是非?鲁国之法,鲁人有赎人臣妾于诸侯,皆受金于府,子贡赎人而不受金。孔子闻而恶之曰:「赐失之矣。夫圣人举事,可以移风易俗,而教道可施于百姓,非独适己之行也。今鲁国富者寡而贫者众,受金则为不廉,何以相赎乎?自今以后,不复赎人于诸侯矣。」

子路拯人于\xpinyin*{溺},其人谢之以牛,子路受之。孔子喜曰:「自今鲁国多拯人于溺矣。」自俗眼观之,子贡不受金为优,子路之受牛为劣;孔子则取由而\xpinyin*{黜}赐焉。乃知人之为善,不论现行而论流弊;不论一时而论久远;不论一身而论天下。现行虽善,其流足以害人;则似善而实非也;现行虽不善,而其流足以济人,则非善而实是也。然此就一节论之耳。他如非义之义,非礼之礼,非信之信,非慈之慈,皆当抉择。

何谓偏正?昔吕文\xpinyin*{懿}公,初辞相位,归故里,海内仰之,如泰山北斗。有一乡人,醉而詈之,吕公不动,谓其仆曰:「醉者勿与较也。」闭门谢之。逾年,其人犯死刑入狱。吕公始悔之曰:「使当时稍与计较,送公家责治,可以小惩而大戒;吾当时只欲存心于厚,不谓养成其恶,以至于此。」此以善心而行恶事者也。

又有以恶心而行善事者。如某家大富,值岁荒,穷民白昼抢\xpinyin*{粟}于市;告之县,县不理,穷民愈肆,遂私执而困辱之,众始定;不然,几乱矣。故善者为正,恶者为偏,人皆知之;其以善心行恶事者,正中偏也;以恶心而行善事者,偏中正也;不可不知也。

何谓半满?易曰:「善不积,不足以成名;恶不积,不足以灭身。」书曰:「商罪贯盈,如贮物于器。」勤而积之,则满;懈而不积,则不满。此一说也。

昔有某氏女入寺,欲施而无财,止有钱二文,捐而与之,主席者亲为忏悔;及后入宫富贵,携数千金入寺舍之,主僧惟令其徒回向而已。

因问曰:「吾前施钱二文,师亲为忏悔,今施数千金,而师不回向,何也?」

曰:「前者物虽薄,而施心甚真,非老僧亲忏,不足报德;今物虽厚,而施心不若前日之切,令人代忏足矣。」 此千金为半,而二文为满也。

钟离授丹于吕祖,点铁为金,可以济世。

吕问曰:「终变否?」

曰:「五百年后,当复本质。」

吕曰:「如此则害五百年后人矣,吾不愿为也。」

曰:「修仙要积三千功行,汝此一言,三千功行已满矣。」

此又一说也。

又为善而心不着善,则随所成就,皆得圆满。心着于善,虽终身勤励,止于半善而已。譬如以财济人,内不见己,外不见人,中不见所施之物,是谓三轮体空,是谓一心清净,则斗粟可以种无涯之福,一文可以消千劫之罪,倘此心未忘,虽黄金万\xpinyin*{镒},福不满也。此又一说也。

何谓大小?昔卫仲达为馆职,被摄至冥司,主者命吏呈善恶二录,比至,则恶录盈庭,其善录一轴,仅如\xpinyin*{箸}而已。索秤称之,则盈庭者反轻,而如箸者反重。

仲达曰:「某年未四十,安得过恶如是多乎?」

曰:「一念不正即是,不待犯也。」

因问轴中所书何事?

曰:「朝廷尝兴大工,修三山石桥,君上疏\xpinyin*{谏}之,此疏稿也。」

仲达曰:「某虽言,朝廷不从,于事无补,而能有如是之力。」

曰:「朝廷虽不从,君之一念,已在万民;向使听从,善力更大矣。」

故志在天下国家,则善虽少而大;苟在一身,虽多亦小。

何谓难易?先儒谓克己须从难克处克将去。夫子论为仁,亦曰先难。必如江西舒翁,舍二年仅得之束修,代偿官银,而全人夫妇;与邯郸张翁,舍十年所积之钱,代完赎银,而活人妻子,皆所谓难舍处能舍也。如镇江\xpinyin*{靳}翁,虽年老无子,不忍以幼女为妾,而还之邻,此难忍处能忍也;故天降之福亦厚。凡有财有势者,其立德皆易,易而不为,是为自暴。贫贱作福皆难,难而能为,斯可贵耳。

随缘济众,其类至繁,约言其纲,大约有十:第一,与人为善;第二,爱敬存心;第三,成人之美;第四,劝人为善;第五,救人危急;第六,兴建大利;第七,舍财作福;第八,护持正法;第九,敬重尊长;第十,爱惜物命。

何谓与人为善?昔舜在雷泽,见渔者皆取深潭厚泽,而老弱则渔于急流浅滩之中,\xpinyin*{恻}然哀之,往而渔焉;见争者皆匿其过而不谈,见有让者,则\xpinyin*{揄}扬而取法之。期年,皆以深潭厚泽相让矣。夫以舜之明哲,岂不能出一言教众人哉?

乃不以言教而以身转之,此良工苦心也。

吾辈处末世,勿以己之长而盖人;勿以己之善而形人;勿以己之多能而困人。 收敛才智,若无若虚;见人过失,且涵容而掩覆之。一则令其可改,一则令其有所顾忌而不敢纵,见人有微长可取,小善可录,翻然舍己而从之;且为艳称而广述之。凡日用间,发一言,行一事,全不为自己起念,全是为物立则;此大人天下为公之度也。

何谓爱敬存心?君子与小人,就形迹观,常易相混,惟一点存心处,则善恶悬绝,判然如黑白之相反。故曰:君子所以异于人者,以其存心也。君子所存之心,只是爱人敬人之心。盖人有亲疏贵贱,有智愚贤不肖;万品不齐,皆吾同胞,皆吾一体,孰非当敬爱者?爱敬众人,即是爱敬圣贤;能通众人之志,即是通圣贤之志。何者?圣贤之志,本欲斯世斯人,各得其所。吾合爱合敬,而安一世之人,即是为圣贤而安之也。

何谓成人之美?玉之在石,\xpinyin*{抵掷}则瓦\xpinyin*{砾},追\xpinyin*{琢}则\xpinyin*{圭璋};故凡见人行一善事,或其人志可取而资可进,皆须诱\xpinyin*{掖}而成就之。或为之奖借,或为之维持;或为白其诬而分其谤;务使成立而后已。

大抵人各恶其非类,乡人之善者少,不善者多。善人在俗,亦难自立。且豪杰\xpinyin*{铮铮},不甚修形迹,多易指摘;故善事常易败,而善人常得谤;惟仁人长者,匡直而辅翼之,其功德最宏。

何谓劝人为善?生为人类,孰无良心?世路\xpinyin*{役役},最易没溺。凡与人相处,当方便提撕,开其迷惑。譬犹长夜大梦,而令之一觉;譬犹久陷烦恼,而拔之清凉,为惠最\xpinyin*{溥}。韩愈云:「一时劝人以口,百世劝人以书。」较之与人为善,虽有形迹,然对证发药,时有奇效,不可废也;失言失人,当反吾智。

何谓救人危急?患难颠沛,人所时有。偶一遇之,当如\xpinyin*{痌癏}之在身,速为解救。或以一言伸其屈抑;或以多方济其颠连。崔子曰:「惠不在大,赴人之急可也。」盖仁人之言哉。

何谓兴建大利?小而一乡之内,大而一邑之中,凡有利益,最宜兴建;或开渠导水,或筑堤防患;或修桥梁,以便行旅;或施茶饭,以济饥渴;随缘劝导,协力兴修,勿避嫌疑,勿辞劳怨。

何谓舍财作福?释门万行,以布施为先。所谓布施者,只是舍之一字耳。达者内舍六根,外舍六尘,一切所有,无不舍者。苟非能然,先从财上布施。世人以衣食为命,故财为最重。吾从而舍之,内以破吾之\xpinyin*{悭},外以济人之急;始而勉强,终则泰然,最可以荡\xpinyin*{涤}私情,祛除执\xpinyin*{吝}。

何谓护持正法?法者,万世生灵之眼目也。不有正法,何以参赞天地?何以裁成万物?何以脱尘离缚?何以经世出世?故凡见圣贤庙貌,经书典籍,皆当敬重而修\xpinyin*{饬}之。至于举扬正法,上报佛恩,尤当勉励。

何谓敬重尊长?家之父兄,国之君长,与凡年高,德高,位高,识高者,皆当加意奉事。在家而奉侍父母,使深爱婉容,柔声下气,习以成性,便是和气格天之本。出而事君,行一事,毋谓君不知而自\xpinyin*{恣}也。刑一人,毋谓君不知而作威也。事君如天,古人格论,此等处最关阴德。试看忠孝之家,子孙未有不绵远而昌盛者,切须慎之。

何谓爱惜物命?凡人之所以为人者,惟此\xpinyin*{恻}隐之心而已;求仁者求此,积德者积此。周礼,「孟春之月,牺牲\xpinyin*{毋用牝}。」孟子谓君子远\xpinyin*{庖}厨,所以全吾恻隐之心也。故前辈有四不食之戒,谓闻杀不食,见杀不食,自养者不食,专为我杀者不食。学者未能断肉,且当从此戒之。

渐渐增进,慈心愈长,不特杀生当戒,蠢动含灵,皆为物命。求丝煮茧,锄地杀虫,念衣食之由来,皆杀彼以自活。故暴\xpinyin*{殄}之\xpinyin*{孽},当与杀生等。至于手所误伤,足所误践者,不知其几,皆当委曲防之。古诗云:「爱鼠常留饭,怜蛾不点灯。」何其仁也!

善行无穷,不能\xpinyin*{殚}述;由此十事而推广之,则万德可备矣。

\chapter{第四篇\ 谦德之效}

易曰:「天道亏盈而益谦;地道变盈而流谦;鬼神害盈而福谦;人道恶盈而好谦。」是故谦之一卦,六\xpinyin*{爻}皆吉。

书曰:「满招损,谦受益。」予屡同诸公应试,每见寒士将达,必有一段谦光可掬。

辛未(公元1571年)计\xpinyin*{偕},我嘉善同\xpinyin*{袍}凡十人,惟丁敬宇宾,年最少,极其谦虚。

予告费锦坡曰:「此兄今年必第。」

费曰:「何以见之?」

予曰:「惟谦受福。兄看十人中,有恂\xpinyin*{恂款}款,不敢先人,如敬宇者乎?有恭敬顺承,小心谦畏,如敬宇者乎?有受侮不答,闻谤不辩,如敬宇者乎?人能如此,即天地鬼神,犹将佑之,岂有不发者?」

及开榜,丁果中式。

丁丑(公元1577年)在京,与冯开之同处,见其虚己敛容,大变其幼年之习。李\xpinyin*{霁}岩直谅益友,时面攻其非,但见其平怀顺受,未尝有一言相报。予告之曰:「福有福始,祸有祸先,此心果谦,天必相之,兄今年决第矣。」已而果然。

赵裕峰,光远,山东冠县人,童年举于乡,久不第。其父为嘉善三尹,随之任。慕钱明吾,而执文见之,明吾悉抹其文,赵不惟不怒,且心服而速改焉。明年,遂登第。

壬辰岁(公元1592年),予入\xpinyin*{觐},\xpinyin*{晤}夏建所,见其人气虚意下,谦光逼人,归而告友人曰:「凡天将发斯人也,未发其福,先发其慧;此慧一发,则浮者自实,肆者自敛;建所温良若此,天启之矣。」及开榜,果中式。

江阴张畏岩,积学工文,有声艺林。甲午(公元1594年),南京乡试,寓一寺中,揭晓无名,大骂试官,以为眯目。时有一道者,在傍微笑,张\xpinyin*{遽}移怒道者。道者曰:「相公文必不佳。」

张怒曰:「汝不见我文,乌知不佳?」

道者曰:「闻作文,贵心气和平,今听公骂詈,不平甚矣,文安得工?」

张不觉屈服,因就而请教焉。

道者曰:「中全要命;命不该中,文虽工,无益也。须自己做个转变。」

张曰:「既是命,如何转变?」

道者曰:「造命者天,立命者我;力行善事,广积阴德,何福不可求哉?」

张曰:「我贫士,何能为?」

道者曰:「善事阴功,皆由心造,常存此心,功德无量,且如谦虚一节,并不费钱,你如何不自反而骂试官乎?」

张由此折节自持,善日加修,德日加厚。丁酉(公元1597年),梦至一高房,得试录一册,中多缺\xpinyin{行}{hang2}。问旁人, 曰:「此今科试录。」

问:「何多缺名?」

曰:「科第阴间三年一考较,须积德无咎者,方有名。如前所缺,皆系旧该中式,因新有薄\xpinyin*{行}而去之者也。」

后指一\xpinyin{行}{hang2}云:「汝三年来,持身颇慎,或当补此,幸自爱。」是科果中一百五名。

由此观之,举头三尺,决有神明;趋吉避凶,断然由我。须使我存心制\xpinyin*{行},毫不得罪于天地鬼神,而虚心屈己,使天地鬼神,时时怜我,方有受福之基。彼气盈者,必非远器,纵发亦无受用。稍有识见之士,必不忍自狭其量,而自拒其福也,况谦则受教有地,而取善无穷,尤修业者所必不可少者也。

古语云:「有志于功名者,必得功名;有志于富贵者,必得富贵。」人之有志,如树之有根,立定此志,须念念谦虚,尘尘方便,自然感动天地,而造福由我。今之求登科第者,初未尝有真志,不过一时意兴耳;兴到则求,兴\xpinyin*{阑}则止。

孟子曰:「王之好\xpinyin{乐}{yue4}甚,齐其庶几乎?」予于科名亦然。

\part{译文}
\chapter{第一篇\ 立命之学}
所谓“立命”,就是我要创造命运,而不是让命运来束缚我。本篇立命之学,就是讨论立命的学问,讲解立命的道理。袁了凡先生将自己所经历,所见到改造命运种种的考验,告诉他的儿子;要袁天启不被命运束缚住,并且应竭力行善,“勿以善小而不为”;也必须努力断恶,“勿以恶小而为之”;如此,则一定可以改变自己的命运,所谓“断恶修善”,“灾消福来”,这是改造命运的原理。

【\kai{千人千般命呀!命命不相同,明朝袁了凡,本来命普通,遇到孔先生,命都被算中;短命绝后没功名,前世业障真不轻,庸庸碌碌二十年,一生命数被算定,云谷禅师来开示,了凡居士才转命呀!才转命。}】

我童年的时候父亲就去逝了,母亲要我放弃学业,不要去考功名,改学医,并且说:学医可以赚钱养活生命,也可以救济别人。并且医术学得精,可以成为名医,这是你父亲从前的心愿。

后来我在慈云寺,碰到了一位老人,相貌非凡,一脸长须,看起来飘然若仙风道骨,我就很恭敬地向他行礼。这位老人向我说:你是官场中的人,明年就可以去参加考试,进学宫了,为何不读书呢?

我就把母亲叫我放弃读书去学医的缘故告诉他。并且请问老人的姓名,是那里人,家住何处;老人回答我说:我姓孔,是云南人,宋朝邵康节先生所精通的皇极数,我得到他的真传。照注定的数来讲,我应该把这个皇极数传给你。

因此,我就领了这位老人到我家,并将情形告诉母亲。母亲要我好好的待他。并且说:这位先生既然精通命数的道理,就请他替你推算推算,试试看,究竟灵不灵。

结果孔先生所推算的,虽然是很小的事情,但是都非常的灵验。我听了孔先生的话,就动了读书的念头,和我的表哥沈称商量。表哥说:我的好朋友郁海谷先生在沈友夫家里开馆,收学生读书。我送你去他那里寄宿读书,非常方便。于是我便拜了郁海谷先生为老师。孔先生有一次替我推算我命里所注定的数;他说:在你没有取得功名做童生时,县考应该考第十四名,府考应该考第七十一名,提学考应该考第九名。

到了明年,果然三处的考试,所考的名次和孔先生所推算的一样,完全相符。孔先生又替我推算终生的吉凶祸福。他说:那一年考取第几名,那一年应当补廪生,那一年应当做贡生,等到贡生出贡后,在某一年,应当选为四川省的一个县长,在做县长的任上三年半后,便该辞职回家乡。到了五十三岁那年八月十四日的丑时,就应该寿终正寝,可惜你命中没有儿子。

这些话我都一一的记录起来,并且牢记在心中。从此以后,凡是碰到考试,所考名次先后,都不出孔先生预先所算定的名次。唯独算我做廪生所应领的米,领到九十一石五斗的时候才能出贡。那里知道我吃到七十一石米的时候,学台屠宗师(学台:相当于现在的教育厅长)他就批准我,补了贡生。我私下就怀疑孔先生所推算的,有些不灵了。

后来果然被另外一位代理的学台杨宗师驳回,不准我补贡生。直到丁卯年,殷秋溟宗师看见我在考场中的‘备选试卷’没有考中,替我可惜,并且慨叹道:这本卷子所做的五篇策,竟如同上给皇帝的奏折一样。像这样有大学问的读书人,怎么可以让他埋没到老呢?

于是他就吩咐县官,替我上公事到他那里,准我补了贡生,经过这番的波折,我又多吃了一段时间的廪米,算起来连前所吃的七十一石,恰好补足,总计是九十一石五斗。我因为受到了这番波折,就更相信:一个人的进退功名浮沉,都是命中注定。而走运的迟或早,也都有一定的时候,所以一切都看得淡,不去追求了。

等我当选了“贡生”,按照规定,要到北京的国家大学去读书。所以我在京城里住了一年。一天到晚,静坐不动,不说话,也不转动念头。凡是文字,一概都不看。到了己巳年,回到南京的国家大学读书,在没有进国家大学以前,先到栖霞山去拜见云谷禅师,他是一位得道的高僧。

我同禅师面对面,坐在一间禅房里,三天三夜,连眼睛都没有闭。云谷禅师问我说:凡是一个人,所以不能够成为圣人,只因为妄念,在心中不断地缠来缠去;而你静坐三天,我不曾看见你起一个妄念,这是什么缘故呢?

我说:我的命被孔先生算定了,何时生,何时死,何时得意,何时失意,都有个定数,没有办法改变。就是要胡思乱想得到什么好处,也是白想;所以就老实不想,心里也就没有什么妄念了。云谷禅师笑道:我本来认为你是一个了不得的豪杰,那里知道,你原来只是一个庸庸碌碌的凡夫俗子。

我听了之后不明白,便请问他此话怎讲?云谷禅师说道:一个平常人,不能说没有胡思乱想的那颗意识心;既然有这一颗一刻不停的妄心在,那就要被阴阳气数束缚了;既被阴阳气数束缚,怎么可说没有数呢?虽说数一定有,但是只有平常人,才会被数所束缚住。若是一个极善的人,数就拘他不住了。

因为极善的人,尽管本来他的命数里注定吃苦;但是他做了极大的善事,这大善事的力量,就可以使他苦变成乐,贫贱短命,变成富贵长寿。

而极恶的人,数也拘他不住。因为极恶的人,尽管他本来命中注定要享福,但是他如果做了极大的恶事,这大恶事的力量,就可以使福变成祸,富贵长寿变成为贫贱短命。

你二十年来的命都被孔先生算定了,不曾把数转动一分一毫,反而被数把你给拘住了。一个人会被数拘住,就是凡夫,这样看来,你不是凡夫,是什么呢?

我问云谷禅师说:照你说来,究竟这个数,可以逃得过去么?禅师说:命由我自己造,福由我自己求;我造恶就自然折福;我修善,就自然得福。从前各种诗书中所说,实在是的的确确,明明白白的好教训。我们佛经里说:一个人要求富贵就得富贵,要求儿女就得儿女,要求长寿就得长寿。

只要做善事,命就拘他不住了。因为说谎是佛家的大戒,那有佛菩萨还会乱说假话,欺骗人的呢?

我听了以后,心里还是不明白,又进一步问说;孟子曾说:凡是求起来,就可以得到的,这是说在我心里可以做得到的事情。

若是不在我心里的事,那么怎能一定求得到呢?譬如说道德仁义,那全是在我心里的,我立志要做一个有道德仁义的人,自然我就成为一个有道德仁义的人,这是我可以尽力去求的。若是功名富贵,那是不在我心里头的,是在我身外的,要别人肯给我,我才可以得到。倘若旁人不肯给我,我就没法子得到,那么我要怎样才可以求到呢?云谷禅师说:孟子的话不错,但是你解释错了。你没看见六祖慧能大师说:所有各种的福田,都决定在各人的心里。福离不开心,心外没有福田可寻,所以种福种祸,全在自己的内心。只要从心里去求福,没有感应不到的!

能向自己心里去求,那就不只是心内的道德仁义,可以求得,就是身外的功名富贵,也可以求到,所以叫做内外双得。换句话说,为了种福田而求仁求义,求福,求禄,是必有所得的。

一个人命里若有功名富贵,就是不求,也会得到;若是命里没有功名富贵,就算是用尽了方法,也求不到的。

所以一个人,若不能自己检讨反省,而只是盲目地向外面追求名利福寿;但得到得不到,还是听天由命,自己毫无把握。这就合了孟子所说,求之有道,得之有命的两句话了。

要知道纵然得到,究竟还是命里本来就有的,并不是自己求的效验,所以可以求到的,才去求,求不到的,就不必去乱求。

倘若你一定要求,那不但身外的功名富贵求不到,而且因为过份的乱求,过份的贪得,为求而不择手段,那就把心里本来有的道德仁义,也都失掉了,那岂不是内外双失么?所以乱求是毫无益处的。

【\kai{求富贵呀得富贵,求儿女呀得儿女,求长寿呀得长寿,没有什么求不到呀,求不到!只要做好事,从心里去求,心就是福田呀,千万别乱求;心就是福田呀,千万别乱求。}】

云谷禅师接著再问我说:孔先生算你终身的命运如何?

我就把孔先生算我,某年考的怎么样,某年有官做,几岁就要死的话详详细细的告诉他。云谷禅师说:你自己想想,你应该考得功名么?应该有儿子么?我反省过去所作所为,想了很久才说:我不应该考得功名,也不应该有儿子。因为有功名的人,大多有福相。

我的相薄,所以福也薄。又不能积功德积善行,成立厚福的根基。并且我不能忍耐,担当琐碎繁重的事情。别人有些不对的地方,也不能包容。因为我的性情急燥,肚量窄小。有时候我还自尊自大,把自己的才干、智力、去盖过别人。心里想怎样就怎么做,随便乱谈乱讲。像这样种种举动,都是薄福的相,怎么能考得功名呢!

喜欢干净,本是好事;但是不可过分,过分就成怪脾气了。所以说越是不清洁的地方,越会多生出东西来。相反地,很清洁的水反而养不住鱼。

我过分地喜欢清洁,就变得不近人情,这是我没有儿子的第一种缘故。

天地间,要靠温和的日光,和风细雨的滋润,才能生长万物。我常常生气发火,没有一点和育之气,怎么会生儿子呢?这是我没有儿子的第二种缘故。

仁爱,是生生的根本,若是心怀残忍,没有慈悲;就像果子一样,没有果仁,怎么会长出果树呢?所以说,忍是不会生养的根;我只知道爱惜自己的名节,不肯牺牲自己,去成全别人,积些功德,这是我没有儿子的第三种缘故。

说话太多容易伤气,我又多话,伤了气,因此身体很不好,那里会有儿子呢?这是我没有儿子的第四种缘故。

人全靠精气神三种才能活命;我爱喝酒,酒又容易消散精神;一个人精力不足,就算生了儿子,也是不长寿的,这是我没有儿子的第五种缘故。

一个人白天不该睡觉,晚上又不该不睡觉;我常喜欢整夜长坐,不肯睡,不晓得保养元气精神,这是我没有儿子的第六种缘故。其它还有许多的过失,说也说不完呢!云谷禅师说:岂只是功名不应该得到,恐怕不应该得的事情,还多著哩!

当知有福没福,都是由心造的。有智慧的人,晓得这都是自作自受;糊涂的人,就都推到命运头上去了。

譬如这个世上能够拥有千金产业的,一定是享有千金福报的人;能够拥有一百金产业的,一定是享有一百金福报的人;应该饿死的,一定是应该受饿死报应的人。比如说善人积德,上天就加多他应受的福。恶人造孽,上天就加多他应得的祸。上天不过就他本来的质地上,加重一些罢了,并没有一丝毫别的意思。

接下来这段是云谷禅师借俗人之见,来劝了凡先生努力积德行善。

就像生儿子,也是看下的种怎样,种下的很厚,结的果也厚。种下得薄,结的也薄。譬如一个人,积了一百代的功德,就一定有一百代的子孙,来保住他的福。积了十代的功德,就一定有十代的子孙,来保住他的福。积了三代或者两代的功德,就一定有三代或者两代的子孙,来保住他的福。至于那些只享了一代的福,到了下一代,就绝后的人;那是他功德极薄的缘故,恐怕他的罪孽,还积得不少哩!你既然知道自己的短处,那就应该把你一向不能得到功名,和没有儿子的种种福薄之相,尽心尽力改得干干净净。一定要积德,一定要对人和气慈悲,一定要替人包含一切,而且要爱惜自己的精神。

从前的一切一切,譬如昨日,已经死了;以后的一切一切,譬如今日,刚刚出生;能够做到这样,就是你重新再生了一个义理道德的生命了。我们这个血肉之躯,尚且还有一定的的数;而义理的、道德的生命,那有不能感动上天的道理?书经太甲篇上面说道:上天降给你的灾害,或者可以避开;而自己若是做了孽,就要受到报应,不能愉快心安地活在世间上了。

诗经上也讲:人应该时常想到自己的所作所为,合不合天道。很多福报,不用求,自然就会有了。因此,求祸求福,全在自己。

【\kai{书经说:天作孽呀,犹可违呀犹可违,自作孽呀,不可活呀,不可活;诗经上也说:常常想自己,所做跟所为,合不合天道,求祸与求福,全在你自己呀!全在你自己。}】

孔先生算你,不得功名,命中无子,虽然说是上天注定,但是还是可以改变。你只要将本来就有的道德天性,扩充起来,尽量多做一些善事,多积一些阴德,这是你自己所造的福,别人要抢也抢不去,那有可能享受不到呢?

易经上也有为一些宅心仁厚、有道德的人打算,要往吉祥的那一方去,要避开凶险的人,凶险的事,凶险的地方。

如果说命运是一定不能改变的,那末吉祥又何处可以趋,凶险又那里可以避免呢?易经开头第一章就说:经常行善的家庭,必定会有多余的福报,传给子孙;这个道理,你真的能够相信吗?

我相信云谷禅师的话,并且向他拜谢,接受他的指教;同时把从前所做的错事,所犯的罪恶,不论大小轻重,到佛前去,全部说出来;并且做了一篇文字,先祈求能得到功名,还发誓要做三千件的善事,来报答天地祖先生我的大恩大德。云谷禅师听我立誓要做三千件的善事,就拿了功过格给我看。叫我照著功过格所订的方法去做,所做的事,不论是善是恶,每天都要记在功过格上,善的事情就记在功格下面,恶的事情就记在过格下面。

不过做了恶事,还要看恶事的大小,把已经记的功来减除。并且还教我念准提咒,更加上了一重佛的力量,希望我所求的事,一定会有效应。云谷禅师又对我说:有一种画符篠的专家曾说:一个人如果不会画符,是会被鬼神耻笑的。

画符有一种秘密的方法传下来,只是不动念头罢了。当执笔画符的时候,不但不可以有不正的念头,就是正当的念头,也要一齐放下。把心打扫得干干净净,没有一丝杂念,因为有了一丝的念头,心就不清净了。到了念头不动,用笔在纸上点一点,这一点就叫混沌开基,因为完整的一道符,都是从这一点开始画起,所以这一点是符的根基所在。

从这一点开始一直到画完整个符,若没起一些别的念头,那么这道符,就很灵验。不但画符不可夹杂念头,凡是祷告上天,或者是改变命运,都要从没有妄念上去用工夫,这样才能感动上天。孟子讲立命的道理说道:短命和长寿没有分别。乍听之下会觉得奇怪?因为短命和长寿相反,而且完全不同,怎样说是一样呢?要晓得在一个妄念都完全没有时,就如同婴儿在胎胞里面的时候,那晓得短命和长寿的分别呢?

等到出了娘胎,渐渐有了知识,有了分别的心;这时,前生所造的种种善业恶业,都要受报应了,那也就有短命和长寿的分别了。

因此,命运是自己造的。如果把立命这两个字细分来讲,那末富和贫要看得没有两样,不可以富的仗著有钱有势,随便乱来,穷的也不可以自暴自弃去做坏事,尽管穷,仍然应该安分守己的做好人;能够这样,才可以把本来贫穷的命,改变成富贵的命。本来富贵的命,改变成更加富贵,或者是富贵得更长久。穷与通,要看得是没有两样,不发达的人,不可因为自己不得志,就不顾一切,随便荒唐;发达的人,也不可仗势欺人,造种种的罪业,越是得意,越是要为善去恶,广种福田。

能够这样,才可以把本来穷苦的命,改变成发达的命,本来发达的命,就会更加发达了。短命和长寿,要看得没有两样,不可说我短命;不久就死了,就趁还活著的时候,随便做恶事,糟蹋自己。要晓得既然已生成短命,就更加应该做好人,希望来生不要再短命,这一生或许也可以把寿命延长一些哦!

命中长寿的人,不要认为自己有得活,就拼命造孽,做奸犯科,犯邪淫。要晓得长寿得来不易,更应该做好人,才可以保住他的长寿呀。能够明白这种道理,才可以把本来短的命变成长寿,本来长寿的命,更加长寿健康。人生在这个世界上,只有这生与死的关系最为重大,所以短命同了长寿,就是最重大的事情。既然说到这最重大的短命同了长寿,那末此外一切顺境,富有和发达;逆境,贫穷和不发达,都可以包括在内了。

孟子讲立命的学问,只讲到短命和长寿,并没讲到富和贫,发达和不发达,就是这个道理。

接著云谷禅师又告诉我说:孟子所说的“修身以俟之”这句话,是说:自己要时时刻刻修养德行,不要做半点过失罪恶。至于命能不能改变,那是积德的事,求天的事。

说到修字,那么身上有一些些过失罪恶,就应该像治病一样,把过失罪恶要完全去掉。讲到俟,要等到修的功夫深了,命自然就会变好,不可以有一丝一毫的非份之想,也不可以让心里的念头乱起乱灭,都要完全把它斩掉断绝,能够做到这种地步,已经是达到先天不动念头的境界了。到了这种功夫,那就是世间受用的真正学问。

云谷禅师接著又说:平常时一般人的行为,都是根据念头转的,凡是有心而为的事,不能算是自然,不著痕迹。你现在还不能做到不动心的境界,你若能念准提咒,不必用心去记或数遍数,只要一直念下去,不要间断。念到极熟的时候,自然就会口里在念,自己不觉得在念,这叫做持中不持;在不念的时候,心里不觉的仍在念,这叫做不持中持;念咒能念到这样,那就我、咒、念打成了一片,自然不会有杂念进来,那末念的咒,也就没有不灵验的了。但是这种功夫,一定要透过实践,才能领会到的。

我起初的号叫做学海,但是自从那一天起就改号叫做了凡;因为我明白立命的道理,不愿意和凡夫一样。把凡夫的见解,完全扫光,所以叫做了凡。

从此以后,就整天小心谨慎,自己也觉得和从前大不相同。从前尽是糊涂随便,无拘无束;到了现在,自然有一种小心谨慎,战战兢兢戒慎恭敬的景象。

虽然是在暗室无人的地方,也常恐怕得罪天地鬼神。碰到讨厌我,毁谤我的,我也能够安然的接受,不与旁人计较争论了。从我见了云谷禅师的第二年,到礼部去考科举。孔先生算我的命,应该考第三名,那知道忽然考了第一名,孔先生的话开始不灵了。孔先生没算我会考中举人,那知道到了秋天乡试,我竟然考中了举人,这都不是我命里注定的,云谷禅师说:命运是可以改造的。这话我更加地相信了。

我虽然把过失改了许多,但是碰到应该做的事情,还是不能一心一意的去做,即使做了,依然觉得有些勉强,不太自然。自己检点反省,觉得过失仍然很多。

例如看见善,虽然肯做;但是还不能够大胆地向前拼命去做。或者是遇到救人时,心里面常怀疑惑,没有坚定的心去救人。自己虽然勉强做善事,但是常说犯过失的话。有时我在清醒的时候,还能把持住自己,但是酒醉后就放肆了。虽然常做善事,积些功德;但是过失也很多,拿功来抵过,恐怕还不够,光阴常是虚度。从己巳年听到云谷禅师的教训,发愿要做三千件的善事;直到己卯年,经过了十多年,才把三千件的善事做完。

在那个时候,我刚和李渐庵先生,从关外回来关内,没来得及把所做的三千件善事回向。到了庚辰年,我从北京回到了南方,方才请了性空、慧空、两位有道的大和尚,借东塔禅堂完成了这个回向的心愿。到这时候,我又起了求生儿子的心愿,也许下了三千件善事的大愿。到了辛巳年,生了你,取名叫天启。

我每做了一件善事,随时都用笔记下来;你母亲不会写字,每做一件善事,都用鹅毛管,印一个红圈在日历上,或是送食物给穷人,或买活的东西放生,都要记圈。有时一天多到十几个红圈呢!也就是代表一天做了十几件善事。

像这样到了癸未年的八月,三千条善事的愿,方才做满。又请了性空和尚等,在家里做回向。到那年的九月十三日,又起求中进士的愿,并且许下了做一万条善事的大愿。到了丙戌年,居然中了进士,吏部就补了我宝坻县县长的缺。我做宝坻县的县长时,准备了一本有空格的小册子,这本小册子,我叫它作治心篇。意思就是恐怕自己心起邪思歪念,因此,叫‘治心’二字。

每天早晨起来,坐堂审案的时候,叫家里人拿这本治心篇交给看门的人,放在办公桌上。每天所做的善事恶事,虽然极小,也一定要记在治心篇上。到了晚上,在庭院中摆了桌子,换了官服,仿照宋朝的铁面御史赵阅道,焚香祷告天帝,天天都是如此。你母亲见我所做的善事不多,常常皱著眉头向我说:我从前在家,帮你做善事,所以你所许下三千件善事的心愿,能够做完。现在你许了做一万件善事的心愿,在衙门里没什么善事可做,那要等到什么时候,才能做完呢?

在你母亲说过这番话之后,晚上睡觉我偶然做了一个梦,看到一位天神。我就将一万件善事不易做完的缘故,告诉了天神,天神说:‘只是你当县长减钱粮这件事,你的一万件善事,已经足够抵充圆满了。’

原来宝坻县的田,每亩本来要收银两分三厘七毫,我觉得百姓钱出得太多,所以就把全县的田清理一遍;每亩田应缴的钱粮,减到了一分四厘六毫,这件事情确实是有的;但也觉得奇怪,怎么这事会被天神知道,并且还疑惑,只有这件事情,就可以抵得了一万件善事呢?

那时候恰好幻余禅师从五台山来到宝坻,我就把梦告诉了禅师,并问禅师,这件事可以相信吗?幻余禅师说:做善事要存心真诚恳切,不可虚情假意,企图回报。那末就是只有一件善事,也可以抵得过一万件善事了。况且你减轻全县的钱粮,全县的农民都得到你减税的恩惠,千万的人民因此减轻了重税的痛苦,而获福不少呢!

我听了禅师的话,就立刻把我所得的俸银薪水捐出来,请禅师在五台山替我斋僧一万人,并且把斋僧的功德来回向。

孔先生算我的命,到五十三岁时,应该有灾难。我虽然没祈天求寿,五十三岁那年,我竟然一点病痛都没有。现在已经六十九岁了(多活了十六年)。书经上说:天道是不容易相信的,人的命,是没一定的。又说:人的命没有一定,是要靠自己创造的。

这些话,一点都不假。我由此方知,凡是讲人的祸福,都是自己求来的,这些话实在是圣贤人的话;若是说祸福,都是天所注定的,那是世上庸俗的人所讲的。

【\kai{天道不易信呀,人命没一定,人命没一定呀,要靠自己造;若说祸与福呀,都是天注定,那是凡夫与俗子,而非圣贤说的话呀,说的话!}】

你的命,不知究竟怎样?就算命中应该荣华发达,还是要常常当作不得意想。就算碰到顺当吉利的时候,还是要常常当作不称心,不如意来想。就算眼前有吃有穿,还是要当作没钱用,没有房子住想。就算旁人喜欢你,敬重你,还是要常常小心谨慎,做恐惧想。就算你家世代有大声名,人人都看重,还是要常常当做卑微想。就算你学问高深,还是要常常当做粗浅想。

这六种想法,是从反面来看问题,能够这样虚心,道德自然会增进,福报也自然会增加。

讲到远,应该要想把祖先的德气,传扬开来;讲到近,应当想父母若有过失,要替他们遮盖起来;这里即是说明孟子的‘父为子隐,子为父隐’的大义所在;讲到向上,应该要想报答国家的恩惠;讲到对下,应该要想造一家的福;说到对外,应该要想救济别人的急难;说到对内,应该要想预防自己的邪念和邪想。

这六种想法,都是从正面来肯定问题,能够常常如此的存心,必然能成为正人君子。

一个人必须要每天知道自己有过失,才能天天改过,若是一天不知道自己的过失,就一天安安逸逸的算自己没过失。如果每天都无过可改,就是每天都没有进步;天底下聪明俊秀的人实在不少,然而他们道德上不肯用功去修,事业不能用功去做;就只为了因循两个字,得过且过,不想前进,所以才耽搁了他们的一生。

云谷禅师所教立命的许多话,实在是最精,最深,最真,最正的道理,希望你要细细的研究,还要尽心尽力的去做,千万不可把大好的光阴虚度过。

\chapter{第二篇\ 改过之法}
人,既然不是生下来就是圣人,那里能没有过失呢?孔子说:“过则勿惮改。”

只要有了过失,就不可以怕改。所以袁了凡先生在讲过改造命运的道理方法后,就接著把改过的方法,详细地说出来,教训他的儿子袁天启。这第二篇就是讲改过的方法。小的过失,尚且要改;那末大的罪孽,自然就不会再造了。

在春秋时代,当时各国的高级官吏,常常要从一个人的言语、行为、去加以判断;就可以猜想到这个人可能遭遇到的吉凶祸福,并且没有不灵验的。这可以在左传和国语这几种书上看得到的。大凡吉祥和凶险的预兆,都在心里发出根苗反应出来,虽然根苗是由心里发出来的,但是会表现到全身的四肢上,譬如一个人很厚道,那么他的全身四肢都会显得稳重。一个人刻薄,那么他的全身四肢都会显得轻佻。

一个人凡是偏在厚道的,一定时常得福;偏在刻薄的,一定时常近祸。一般人没有见识,眼光像被一层膜给遮住了,甚么都看不到;就说祸福没有一定,而且是无法预测的。

一个人能够做到极诚实,毫无半点虚假,这个人的心就可以与天心相合了,因此;能够用诚心处人处事,福祸就会自然降临。所以观察一个人,只要看他的行为,都是善的,就可以预知他的福,就会来了。

相反的,观察一个人,只要看他的行为,都是不善的,就可以预知他的祸,就要来了。人若是要得福,要远离灾祸;在没有讲到做善事前,先要把自己的过失改掉。

但改过的方法,第一要发‘羞耻心’。想想古时候的圣贤,和我一样,都是男子汉、大丈夫,为什么他们可以流芳百世,大家还要以他们做为师表榜样;而我为什么这一生就搞得身败名裂呢?

这都是因为自己过份贪图享乐,受到种种坏环境的污染,偷偷做出种种不应该做的事,自己还以为旁人不知道,目无国法,毫无惭愧之心;就这样天天的沉沦下去,同禽兽一样了,自己却还不知道。

世界上,令人可羞可耻的事情,没有比这个更大的了。孟子说:一个人最大的,最要紧的事情就是这个耻字。为什么呢?因为晓得这个耻字,就会把自己的过失尽量改掉,就可以成为圣贤;若不晓得这个耻字,就会放肆乱来失掉人格,便和禽兽相同了。

这些话都是改过的真正秘诀。改过的第二个方法,是要发戒慎恐惧的心。要知道天地鬼神,都在我们的头上。

鬼神和我们不一样,它们什么都看得到,所以鬼神是不容易被欺骗的。我虽然在大家看不到的地方犯错,但是天地鬼神,实际上就像镜子那样的照着我,把我的过失罪恶照得清清楚楚。过失重的,就有种种的灾祸,降到我的身上来;就算过失轻的,也要减损我现在的福报,我怎么能够不怕呢?

不只是像前面所说的而已。就是在自己家里空闲的地方;但神明的监察,仍然是非常的厉害,非常的清楚。

我虽然把过失遮盖得十分秘密,掩饰得十分巧妙;但是在神明看来,我的肺肝,早被看透,马脚全露出来了。到最后还是没有办法欺骗自己,若是被旁人看破,这个人就一文不值了。又怎么可以不时常存著一颗戒慎恐惧的心呢?

这还不只像上面所说的种种呢!一个人只要一口气还在,就算是犯下滔天的罪过,还是可以忏悔改过的。

古时候有个人,作了一辈子的恶事;到他快死的时候,忽然悔悟,发了一个很大的善念,就立刻得到好死。

这就是说,人若是在紧要关头能够转一个非常痛切又勇猛的善念,便可以把百年所积的罪恶洗干净。譬如千年黑暗的山谷,只要有一盏灯照了进去;光到之处,就可以把千年来的黑暗,完全除去了。所以过失不论长久,或者是新犯的;只要能改,就是了不起。

虽然有过失只要改过就好,但是绝对不可以认为犯过可以改,就是常常犯也不要紧,这是万万不可以的。如果是这样,就是有心犯过,罪就更加重了。

并且在这个不清净的世间,是幻灭无常的,我们这个血肉之身,是非常容易死的;只要一口气喘不过来,这个身体,就不是我的了;到那个时候就是想要改,也没法子改了。并且人死了后,什么都带不去;只有这个孽,是一定跟去的。

因此,明的报应,在阳间你要承担千百年的恶名;虽然你有孝顺的儿子,和可爱的孙子,也不能替你洗清恶名;暗的报应,在阴间,还要千百劫的时间,沉沦在地狱受无量无边的大苦。虽然碰到圣人,贤人,佛菩萨也不能救助你,接引你,这样怎么能不怕呢?

第三,一定要发一直向前的勇猛心。一个人之所以有了过失还不肯改,都是因为得过且过,不能振作奋发,堕落退后的缘故。

要知道若是要改过,一定要起劲用力,当下就改,绝对不能够拖延疑惑,也不可以今天等明天,明天等后天,一直拖下去。小的过失,像尖刺戳在肉里,要赶紧挑掉拔掉。大的过失,像毒蛇咬到手指头一样的厉害,要赶紧切掉手指头,不可有丝毫的犹疑延迟的念头;否则蛇毒在身中散开,人就会死。就像易经中的益卦所讲,风起雷动,万物都生长起来,利益是这样的大。这是比喻人若能够改过迁善,其利益是最大的。

【\kai{改过要发心呀!改过要发心。发些什么心呀!发些什么心。第一要发那羞耻心,第二要发那敬畏心,第三要发那勇猛心,具备这三种心,便能有过立即改呀!立即改。}】

一个人改过,如果能具备以上所说的羞耻心,敬畏心,勇猛心这三种心,那么就能有过立刻改了,就像春天的薄冰,碰到太阳光一样,还怕不融化吗?但要改过,有三种方法。一种是从事实上改,一种是从道理上改,一种是从心念上改。

因为用这三种不同的功夫,所以得到的效验,也自然不会一样。现在先从事实上改的这一句,来加以说明。

譬如前天杀了活的东西,今天起禁止不再杀了。前天发了火骂人,今天起禁止不再发火了。这种就是在事情的本身来改错,禁止不再犯的方法。但是勉强压住,不再犯,比自然而然的改,要难百倍。并且这犯过的病根没有去掉,仍在心里。虽然一时勉强压住,终究还是要露出来的,就像东边把它灭了,西边又会冒出来一样,这究竟不是彻底拔除干净的改过方法。

我再把从理上改过的方法加以说明;肯努力改过的人,在他没有禁止做这件事之前,先要明白这事不能做的道理;譬如一个人,所犯的过失在杀生;那么他先应该想到:上天有好生之德,凡是有生命的,都会爱惜生命而且怕死。杀它的生命,来养我的身体,自问心能安吗?而且有些东西,虽已被杀,但是还没有完全死,像鱼和毛蟹之类。在半死半活的时候放进锅子里烧,这样的痛苦,一直要透到骨髓里;你看罪过不罪过呢?而供养自己,就要用各种贵重的,味道好的东西,摆满了一桌。虽然这样地讲究,但是一经吃过,便成渣滓,什么都没有了。要晓得人吃蔬菜素食素汤等等,也吃得饱啊!何必一定要去伤害生命,造杀生的罪孽,减少自己的福报呢?

又想,凡是有血气有生命的东西,都有灵性知觉,既然都有灵性知觉,那么和我都是一样的了,就算是自己不能修到道德极高的地步,使他们都来尊重我,亲近我,像古时候的圣人大舜,还在他种田的时候就有象替他犁田,鸟帮他拔草。又怎能天天伤害生命,使它们与我结仇,恨我到永无尽期呢?能想到这些,那就会面对桌上有血肉,有生命的菜肴,自然觉得伤心而不能下咽了。譬如像前天喜欢发怒,应该想到:人各有各的长处,也各有各的短处;碰到他人短处的地方,按照情理,应该要哀怜他的苦恼,原谅他的短处;若是有人不讲道理冒犯了我,那是错在他,与我有什么关系呢?本来就没什么怒可以发的呀!又想到:天下,绝对没有自以为什么错都没有的英雄豪杰,因为一个人自以为了不起,那是最笨的人。天下也绝对没有怨恨旁人的学问;因为人若是真正有学问,就会更加谦虚;而且能严以责己,宽以待人,那里会怨恨别人呢?所以怨恨别人的人,定无学问。

因此,一个人做事处处不能称心,都是因为自己的道德没修好,功德没修满,感动人的心不够呀!应该都要反过来自我反省检讨。自己有没有对不起他人的地方?

能够这样的存心用功,那么别人毁谤我,反而变成磨炼我,成就我反面的教育场所了。我应该欢欢喜喜地接受别人给我的教训、批评,还有什么怨恨呢?

还有,听到别人说我坏话而能够不生气,尽管坏话说得很厉害,像火光薰天,也不过是像拿火去烧空中,虚空中无物可烧;而火却是终归要熄灭的。若是听到别人说坏话,你就生气;虽然你用尽心思,尽力去辩,结果却像春天的蚕吐丝,把自己束缚住一样;这就是所谓的作茧自缚,自讨苦吃。所以生气不但是无益处,并且还是有害的。

这都是说生气的后果。至于其它种种的过失和罪恶,也都应该依道理,细细去想,像上边所说的种种道理能够明白,那就自然而然地不会犯过失了。怎样叫做从心上改过呢?人的过失,有千千万万种那么多,都是从心上造出来的,我的心不动,就什么事情都不会造出来,那么过失还会从何处生出来呢?凡是读书人,或是喜欢女色,或是喜欢名声,或是喜欢财物,或是喜欢发火;像这样种种的过失,不必要一类一类的去寻求灭过的方法;只要一心一意地发善心,做善事,正的念头出现在前;那末邪的念头,自然就污染不上了。

譬如亮热的太阳当空而照,所有的妖怪,自然会逃避消失了;这就是最精纯而唯一的修心补过的真正诀窍啊!须知道过失全部是由这颗心造的,因此也应该由这颗心上来改;正好像斩除毒树一样,要斩就斩得干净俐落,连根铲除,才不会再长出来;那又何必要一枝一枝的剪,一叶一叶的摘呢?

改过最上最高的方法,还是修心。能修心,就可使心立刻清净。因为犯过失,都是心上动了种种坏念头的缘故。能修心,那末坏念头一动,就自己发觉。自己能发觉,就立刻把心停住不动;心不动,那么坏念头便消失,也就不会再犯了。若是再不能够这样,那么一定要明白,所犯过失的理由,把这种犯过的念头去掉。若是再不能够这样,那么只好碰到犯过时,用勉强压住的方法,来禁止不犯。如果用修心的上等功夫,和明白不可犯过的道理,用打发它去的下等功夫,以及碰到犯过用强压方法禁止的下等功夫;这上下两等的功夫,同时用,也不一定就失算呀!若是坚持只用下等功夫,反而把修心的上等功夫忽略不用,那就是最笨不过的了。

但是发愿改过,也要有助力;明里头,要有真正的益友在你糊涂的时候时常来提醒你;暗里头,要有鬼神替你证明;(像我把自己所犯的过失,做了篇疏文,上告天地鬼神那样。)还要一心一意的虔诚忏悔,从早到晚,从日到夜,绝不放松;像我这样忏悔经过一个七天,两个七天;直到一个月,两个月,三个月....这样忏悔下去,一定会有效验的!

【\kai{改过须发愿,也要有助力,明里头,要有益友来提醒呀!来提醒。暗里头,要有鬼神做证明呀!做证明。还要一心一意虔诚的忏悔呀!虔诚的忏悔。}】

上面所说忏悔过恶的效验是什么呢?譬如你或许觉得精神上很舒服,心中很宽闲;或觉得以往很笨,忽然智慧大开;或是虽然处在烦忙纷乱之际,心中仍清清朗朗,无所不通;或碰到怨家仇人,而能全把恨心火气消除,而心生欢喜;或是在梦里,感觉吐出黑的东西来;这是种种邪念邪思,积成的一种秽气,梦里吐出,那么心地就清净多了。或是梦到古时候的圣贤来提拔我,牵引我,或是梦见自己会飞到虚空中去,逍遥自在;或是梦见各种彩旗以及装饰珍宝的伞盖,这种种少有少见的事情,都是过失消除罪孽灭去的好征兆。但是也不能因为碰到这些好征兆。就自己以为了不起,而阻断了再上进,再努力的途径。

从前春秋时代卫国的贤大夫蘧伯玉在二十岁的时候,已经能时时反醒自己过去的过失,加以检讨,完全改掉了。到了二十一岁的时候,又觉得从前所改的过失,并不彻底;到了二十二岁,再回忆二十一岁时,还像在梦中一般,像这样一年一年的过去,一年一年的逐步改过;直到五十岁那年,还觉得过去的四十九年,都是有过失的。古人对于改过的学问讲究就是像这样的。

【\kai{蘧伯玉呀贤大夫,二十岁时就觉悟,时时反省己过失,年年检讨再检讨;总觉得,自己改过的工夫不彻底呀,不彻底!所以说,改过的学问须讲究呀,须讲究!}】

我们都是平凡人,过失罪恶,就像刺猬身上的刺一样,聚集了满身都是。而回想过去的事,常常像看不到自己有甚么过失,这实在都是因为粗心,不知道自我反省。又像眼睛上长了翳,看不到自己天天在那里犯过呀!但是,一个人的过失,罪恶深重到了相当的地步,也有证据可以看出来;或者是心思混乱塞住,精神萎靡不振,随便甚么事转头就忘记了;或者是不值得烦恼的事,也常常感觉非常的烦恼;或者是见到品德高尚的君子,便觉得难为情,垂头丧气;或者是听到光明正大的道理,反倒觉得不欢喜;或者是有恩惠给别人,对方不领情反而怨恨你;或者是夜里都做些颠颠倒倒的坏梦,甚至语无伦次失掉平常的模样;像这样种种不正常的现象,都是作孽的表现啊!

假使你有上边所说的那种情形,就应该即刻提起精神,奋发向上,把旧的种种过失一齐改掉;而另外开辟一条新的人生大道,希望你千万不可自己耽误自己啊!

\chapter{第三篇\ 积善之方}
上一篇所讲,改过的种种方法,能够把今生的过失改掉,自然好命就不会变成坏命了;但是还不能把坏命变成好命。因为这一生虽然不犯过失造罪孽,但是前世有没有犯过失,造罪孽,却不知道,若是前世已经犯,这一世虽然不再犯;但是前世所犯的罪过,还是要受报应。那么要怎么样做才能使坏命转成好命呢?这不但要改过,还要积善、积德,才可以把前世所造的罪孽消去。善事积多了,自然能转坏命成好命,并且可以证明它的效验!

【\kai{中国积德第一人,就数山东孔圣人,世代子孙都不衰呀!都不衰,七十三代孔德成呀!孔德成。}】

易经上说:积善的家庭,一定会有很多福份喜庆的事。例如,从前姓严的人家,要把他的女儿,许配给孔子的父亲;就将孔家所作的事情,一件一件都提出来;觉得孔家祖先所积的德,多而且长久;所以预知孔家的子孙,将来必定会大发。后来果然生出了孔子。还有,孔子称赞舜的孝,是不平凡的孝顺,孔子说:像舜这样的大孝,不但祖先要享受他的祭祀;并且他的世世代代子孙可以保住他的福德,不会败落。春秋时代的陈国,就是舜传下来的子孙,足以证明舜的后代兴发得相当长久。这都是非常确实的说法啊!

现在我再以过去发生真实的事情,来证明积善的功德。有一位做过少师的人,姓杨名荣,是福建省建宁人。他家世代是以摆渡为生。有一次,雨下得太久,溪水满涨,水势汹涌横冲直撞,把民房都冲失了,被淹死的人顺著水势一直流下来。别的船都去捞取水中漂来的各种财货,只有少师的曾祖父和祖父,专门去救水里漂来的灾民,而财物一件都不捞,乡人都偷笑他们是傻瓜。等到少师的父亲出生后,家道也渐渐的宽裕了。有一位神仙化做道士的模样,向少师的父亲说:你的祖父和父亲,都积了许多阴功,所生的子孙应该发达做大官。可以将你的父亲葬在某一个地方。少师的父亲听了,就照道士所指定的地方,把他的祖父和父亲葬下。这座坟,就是现在大家所知道的白兔坟。后来少师出生了,到了二十岁就中了进士。一直做官,做到三公里面的少师。皇帝还追封他的曾祖父、祖父、父亲,与少师一样的官位。而且少师的后代子孙,都非常兴旺,一直到现在还有许多贤能之士。

【\kai{杨少师呀!杨少师,祖父曾祖积阴德呀!积阴德;大水来了只救人,财物一概都不取,旁人笑伊是傻瓜,谁知傻瓜享大福呀!享大福!}】

浙江宁波人杨自惩,起初在县衙做书办,心地非常厚道;而且守法公平,做事公正;当时的县官,为人严厉方正,有一次偶然打了一个囚犯,一直打到血流到地上,县官还是不息怒;杨自惩就跪下,替囚犯向县官求情,请县官宽谅那个囚犯。县官说:你求情本来没有什么不能放宽的,但是这个囚犯,不守法律,违背道理,不能教人不生气啊!

杨自惩一边叩头一边说:在朝廷中已经没有是非可言了,政治一片黑暗、贪污、腐败,人心散失已经很久了,审问案件若是审出实情,尚且应该替他们伤心,可怜他们不明事理,误蹈法网,不可以因为审出了案情,就欢喜。若是存心欢喜,恐怕会把案件忽略弄错。若是生气,又恐怕犯人受不住打,勉强招认,容易冤枉人。既然欢喜尚且不可,又怎么可以发火呢?

那县官听了杨自惩的话,非常感动,面容立即和缓下来,不再发怒了!讲到杨自惩的家里,是很穷的;但是他虽然穷,别人送他东西,他一概不肯接受。碰到囚犯缺粮,他却常用许多方法去弄一些米来,救济他们。有一天来了几个新的囚犯,没有东西吃,非常的饿,他自己家里刚巧也欠米。若是拿来给囚犯吃,那么自己家人就没得吃了。如果只顾自己吃,那么囚犯又饿得很可怜,没有办法,便同他的妻子商量。他的妻子问他说:犯人从什么地方来的?从杭州来的。沿途熬饿,脸上饿得没有一点血色;就像一种又青又黄的菜色,几乎可以用手捧起来。

因此,两夫妇就把自己所存的一些米,用来煮稀饭给新来的囚犯吃。然后他们生了两个儿子,大的叫做守陈,小的叫做守址,作官一直做到南北吏部侍郎。大孙子做到刑部侍郎。小孙子也做到四川按察使。两个儿子,两个孙子,都是名臣;而当今有两个名人楚亭和德政,都是杨自惩的后代。

【\kai{囚犯苦呀!囚犯苦,即坐监牢又挨饿,心中凄苦谁人知呀!谁人知;杨书办呀心厚道,夫妻同心帮囚犯,积善之家庆有余呀!庆有余。}】

从前明朝英宗正统年间,有一个土匪首领叫作邓茂七,在福建一带造反。福建的读书人和老百姓,跟随他一起造反的很多。皇帝就起用曾经担任都御使的鄞县人张楷,去搜剿他们。张都宪用计策把邓茂七捉住了。后来张都宪又派了福建布政司的一位谢都事,去搜查捉拿剩下来的土匪,捉到就杀;但是谢都事不肯乱杀,怕杀错人。便向各处寻找依附贼党的名册,查出来凡是没有依附贼党,名册里还没有他们姓名的人。就暗中给他们一面白布小旗,约定他们,搜查贼党的官兵到的那一天,把这面白布小旗插在自己家门口,表示是清白的民家,并且禁止官兵不准乱杀。因为有这种措施而避免被杀的人,大约有一万人之多。后来谢都事的儿子谢迁,就中了状元,官做到宰相。而且他的孙子谢丕,也中了探花,就是第三名的进士。

【\kai{将军呀,不乱杀,后世子孙一定发呀!一定发,谢都事,心慈悲,全活万人子孙昌呀!子孙昌!}】

在福建省浦田县的林家,他们的上辈中,有一位老太太喜欢做善事,时常用米粉做粉团给穷人吃。只要有人向她要,她就立刻给,脸上没有表现出一点厌烦的样子。有一位仙人,变作道士,每天早晨向她讨六、七个粉团。老太太每天给他,一连三年,每天都是这样的布施,没有厌倦过,仙人晓得她作善事的诚心,就向她说:我吃了你三年的粉团,要怎样报答你呢?这样吧,你家后面有一块地,若是你死后葬在这块地上,将来子孙有官爵的,就会像一升麻子那样的多。

后来老太太去世了,她的儿子依照仙人的指示,把老太太安葬下去。林家的子孙第一代发科甲的,就有九人。后来世世代代,做大官的人非常多。因此,福建省竟有一句:‘如果没有姓林的人去赴考,就不能发榜。’的传言。意思是讲:林家考试的人多,并且都能考中,所以到发榜,榜上就不会没有姓林的人。表示林家有功名的人很多。

【\kai{林家老太太,喜欢做好事,常常把米粉,做成粉团送人吃呀,送人吃!布施心诚恳,神仙也感动,报答老太太,子孙官爵一大堆呀,一大堆!}】

冯琢庵太史的父亲,当他在县学里做秀才的时候,有一个非常寒冷的冬天清早,在要去县学的路上,碰到一个人倒在雪地里,用手摸摸看,已经几乎快要冻死了。冯老先生马上就把自己穿的皮袍,脱下来替他穿上;并且还扶他到家里,把他救醒。冯老先生救人后,就做了一个梦,梦中见到一位天神告诉他说:你救人一命,是完全出自一片至诚的心来救的,所以我要派韩琦投生到你家,做你的儿子。等到后来琢庵生了,就命名叫作冯琦。因为他是宋朝一个文武全才的贤能宰相,叫作韩琦的人来投胎转世的。

【\kai{冯老爹,心肠好,救人命呀!功德高;诚心诚意救人命,胜过建造七浮屠呀!七浮屠。}】

浙江台州有一个应大猷尚书,壮年的时候在山中读书,夜里头,鬼常聚在一起做鬼叫,来吓唬人,只有应公不怕鬼叫。有一夜,应公听到一个鬼说:有一个妇人,因为丈夫出远门作客,好久没回来,她的公婆判断儿子可能已经死了,所以就逼这个妇人改嫁;但是这个妇人却是要守节,不肯改嫁。所以明天夜里,她要在这里上吊,我可以找到一个替身了。凡是上吊或者是淹死的人,如果没有替身,便无法投生,所以叫替死鬼。

应公听到这些话,动了救人的心,偷偷的把自己的田,卖了四两银子,还马上写了一封假托她丈夫的信;并把银子寄回家的事写在信上说明。这位出外人的父母看了信以后,因为笔迹不像,所以怀疑信是假的。但是后来他们又说:信是可以假的,但是银子不能假呀!一定是儿子很平安,才会把银子寄回来。

他们这样想以后,就不再逼媳妇去改嫁了。后来他们的儿子回来了,这对夫妇就得以保全,像从前新婚时一样,好好的过日子了。隔天晚上,应公又听到那个鬼说:我本来可以找到替身了,那知道被这个秀才坏了我的事啊。

旁边一个鬼说:喂!你为什么不去害死他呢?

那个鬼说:天帝因为这个人心好,有阴德,已经派他去做阴德尚书了,我怎么还能害他呢?

应公听了这两个鬼所讲的话以后,就更加努力,更加发心,善事一天一天去做,功德也一天一天的增加;碰到荒年的时候,每次都捐米谷救人;碰到亲戚有急难,他一定想尽办法帮助人家渡过难关;碰到蛮不讲理的人,或不如意的事,总会反省,责备自己有过失,就心平气和地接受事实。因为应公能够这样做人,所以他的子孙得到功名,官位的,一直到现在还是很多哩!

【\kai{淹死的人呀!吊死的人,都要找替身呀!都要找替身,所以叫做替死鬼呀!替死鬼!}】

江苏省常熟县有一位徐凤竹先生,他的父亲本来就很富有。偶然碰到了荒年,就先把他应收的田租,完全捐掉,做为全县有田的人的榜样。同时又分他自己原有的稻谷,去救济穷人。有一天夜里,他听到有一群鬼在门口唱道:千也不说谎,万也不说谎,徐家秀才,快要做到了举人!

【\kai{千也不说谎,万也不说谎,徐家的秀才,快做举人郎呀!快做举人郎。}】

那些鬼连续不断的呼叫,夜夜不停。这一年,徐凤竹去参加乡试,果然考中了举人。他的父亲因此更加高兴,努力不倦地做善事,积功德;同时又修桥铺路,施斋饭供养出家人;碰到缺米缺衣的人,也接济他们;凡是对别人有好处的事情,无不尽心的去做。后来他又听到鬼在门前唱道:千也不说谎,万也不说谎,徐家举人,做官直做到都堂!结果徐凤竹,官做到了两浙的巡抚。

【\kai{千也不说谎,万也不说谎,徐家的举人,官做到都堂呀!官做到都堂。}】

浙江省嘉兴县有一位姓屠,名叫康僖的人,起初在刑部里做主事的官,夜里就住在监狱里。并且仔细的盘问囚犯,结果发现没罪而被冤枉的,有不少人;但是屠公并不觉得自己有功劳,他秘密地把这件事,上公文告诉了刑部堂官。

后来到了秋审的时候,刑部堂官,把屠公所提供的话,拣些要点,来审问那些囚犯。囚犯们都老老实实的向堂官供认,没有一个不心服的。因此,堂官就把原来冤枉的,因为受刑不住被逼招认的,释放了十多人。

那个时候京里的百姓,都称赞刑部尚书明察秋毫。后来屠公又向堂官上了一份公文说:在天子脚下,尚且有那么多被冤枉的人;那么全国这样大的地方,千千万万的百姓,那会没有被冤枉的人呢?所以应该每五年再派一位减刑官,到各省去细查囚犯犯罪的实情,确实有罪的,定罪也要公平;若是冤枉的,应该翻案重审,减轻或者释放。

尚书就代为上奏皇帝,皇帝也准了他所建议的办法;就派减刑官,到各省去查察,刚巧屠公也派在内。有一天晚上屠公梦见天神告诉他说:你命里本来没有儿子,但是因为你提出减刑的建议,正与天心相合;所以上帝赐给你三个儿子,将来都可以做大官;穿紫色的袍,束金镶的带。这天晚上,屠公的夫人就有了身孕;后来生下了应埙、应坤、应竣三个儿子,果然都作了高官。

【\kai{屠康僖呀!屠康僖,办刑案,有一套,明察秋毫不贪功,平反冤案十多起,建议减刑合天心,命里无子得三子,个个都是做高官呀!做高官。}】

有一位嘉兴人,姓包,名叫凭,号信之。他的父亲做过安徽池州府的太守。生了七个儿子,包凭是最小的。他被平湖县姓袁的人家,招赘做女婿;和我父亲常常来往,交情很深。他的学问广博,才气很高,但是每次考试都考不中。因此他对佛教、道教的学问,很注意研究。

有一天,他向东去卯湖游玩,偶然到了一处乡村的佛寺里,因为寺内房屋坏了,看见观世音菩萨的圣像,露天而立,被雨淋得很湿。当时就打开他的袋子,有十两银子,就拿给这寺里的住持和尚,叫他修理寺院房屋。和尚告诉他说:修寺的工程大,银子少,不够用,没法完工。

因此,他又拿了松江出产的布四匹,再捡竹箱里的七件衣服给和尚。这七件衣服里,有用麻织的料做的夹衣,是新做的;他的佣人要他不要再送了,但是包凭说:只要观世音菩萨的圣像,能够安好,不被雨淋,我就是赤身露体又有甚么关系呢?和尚听后流著眼泪说:施送银两和衣服布匹,还不是件难事,只是这一点诚心,怎么容易得到呀!

后来房屋修好了,包凭就拉著他父亲同游这座佛寺,并且住在寺中。那天晚上,包凭做了一个梦,梦到寺里的护法神,来谢他说:你做了这些功德,你的儿子可以世世代代享受官禄了。后来他的儿子包汴,孙子包柽芳,都中了进士,做到高官。

【\kai{包信之,学问好,才气高,可惜考试都不中;偶到一处乡村里,见佛寺、观音像,风吹雨打露天立,布施心、油然生,难舍能舍心虔诚,感动得和尚流眼泪,梦里护法来道谢,儿孙世世受官禄呀!受官禄。}】

浙江省嘉善县有一个叫做支立的人,他的父亲,在县衙中的刑房当书办。有一个囚犯,因为被人冤枉陷害,判了死罪;支书办很可怜他,想要替他向上面的长官求情,宽免他不死。那个囚犯晓得支书办的好意之后,告诉他的妻子说:支公的好意,我觉得很惭愧,没法子报答;明天请他到乡下来,你就嫁给他,他或者会感念这份情份,那么我就可能有活命的机会了。

他的妻子听了之后,没别的办法,所以就边哭边答应了。到了明天,支书办到了乡下,囚犯的妻子就自己出来劝支书办喝酒,并且把他丈夫的意思,完全告诉了支书办。但是支书办不愿意这样做,不过究竟还是尽了全力替这个囚犯,把案子平反了。后来,囚犯出狱,夫妻两个人一起到支书办家里叩头拜谢说:您这样厚德的人,在近代实在是少有。现在您没有儿子,我有一个女儿,愿意送给您做扫地的小妾。这在情理上是可以说得通的。

支书办听了他的话,就预备了礼物,把这个囚犯的女儿迎娶为妾,后来生了一个儿子叫支立,才二十岁就中了举人的前茅,官做到翰林院的书记,后来支立的儿子叫做支高,支高的儿子叫支禄,都被保荐做州学县里的教官。而支禄的儿子叫支大纶,也考中了进士。

【\kai{支书办,刑房吏,有位囚犯遭冤屈,被判死罪真可怜;支书办,心慈悲,平反冤狱无条件,厚德感动死囚犯呀!死囚犯!}】

以上这十条故事,虽然每人所做的各不相同,不过行的都是一个善字罢了。若是要再精细的加以分类来说,那末做善事;有真的,有假的;有直的,有曲的;有阴的,有阳的;有是的,有不是的;有偏的,有正的;有一半的,有圆满的;有大的,有小的;有难的,有易的。

这种种都各有各的道理,都应该要仔细的辨别。若是做善事,而不知道考究做善事的道理,就自夸自己做善事,做得怎样有功德,那里知道这不是在做善事,而是在造孽。这样做岂不是冤枉,白费苦心,得不到一些益处啊!我现在把上面所说过的,分类来加以说明。怎么叫做真假呢?从前在元朝的时候有几个读书人,去拜见天目山的高僧中峰和尚,问说:佛家讲善恶的报应,像影子跟著身体一样,人到那里,影子也到那里,永远不分离。这是说行善,定有好报,造恶定有苦报,决不会不报的。为什么现在某一个人是行善的,他的子孙反而不兴旺?有某一个人是作恶的,他的家反倒发达得很?那末佛说的报应,倒是没有凭据了。

中峰和尚回答说:平常人被世俗的见解所蒙蔽,这颗灵明的心,没有洗除干净,因此,法眼未开,所以把真的善行反认为是恶的,真的恶行反算它是善的,这是常有的事情;并且看错了,还不恨自己颠颠倒倒,怎么反而抱怨天的报应错了呢?

大家又说:善就是善,恶就是恶,善恶那里会弄得相反呢?

中峰和尚听了之后,便叫他们把所认为是善的,恶的事情都说出来。其中有一个人说:骂人,打人是恶;恭敬人,用礼貌待人是善。

中峰和尚回答说:你说的不一定对喔!

另外一个读书人说:贪财,乱要钱是恶;不贪财,清清白白守正道,是善。

中峰和尚说:你说的也不一定是对喔!

那些读书人,都把各人平时所看到的种种善恶的行为都讲出来,但是中峰和尚都说:不一定全对喔!

那几个读书人,因为他们所说的善恶,中峰和尚都说他们说得不对,所以就请问和尚,究竟怎样才是善?怎样才是恶?

中峰和尚告诉他们说:做对别人有益的事情,是善;做对自己有益的事情,是恶。若是做的事情,可以使别人得到益处,那怕是骂人,打人,也都是善;而有益于自己的事情,那么就是恭敬人,用礼貌待人,也都是恶。所以一个人做的善事,使旁人得到利益的就是公,公就是真了;只想到自己要得到的利益,就是私,私就是假了。并且从良心上所发出来的善行,是真;只不过是照例做做就算了的,是假。还有,为善不求报答,不露痕迹,那么所做的善事,是真;但是为著某一种目的,企图有所得,才去做的善事,是假;像这样的种种,自己都要仔细地考察。

怎样叫做端曲呢?现在的人,看见谨慎不倔强的人,大都称他是善人,而且很看重他;然而古时的圣贤,却是宁愿欣赏志气高,只向前进的人,或者是安份守己,不肯乱来的人。因为这种人,才有担当;有作为,可以教导他,使他上进。

至于那些看起来谨慎小心却是无用的好人,虽然在乡里,大家都喜欢他;但是因为这种人的个性软弱,随波逐流,没有志气,所以圣人一定要说这种人,是伤害道德的贼。这样看来,世俗人所说的善恶观念,分明是和圣人相反。

俗人说是善的,圣人反而说是恶;俗人说是恶的,圣人反而说是善。从这一个观念,推广到各种不同的事情来说,俗人所喜欢的,或者是不喜欢的,完全不同于圣人。那还有不错的吗?天地鬼神庇佑善人报应恶人,他们都和圣人的看法是一样的,圣贤以为是对的,天地鬼神也以为是对的;圣贤以为是错的,天地鬼神也认为是错的,而不和世俗人采取相同的看法。所以凡要积功德,绝对不可以被耳朵所喜欢的声音,眼睛所喜欢的景象所利用,而跟著感觉在走;必须要从起心动念隐微的地方,将自己的心,默默地洗涤清净,不可让邪恶的念头,污染了自己的心。

所以全是救济世人的心,是直;如果存有一些讨好世俗的心,就是曲。全是爱人的心,是直;如果有一丝一毫对世人怨恨不平的心,就是曲;全是恭敬别人的心,就是直;如果有一丝玩弄世人的心,就是曲。这些都应该细细的去分辨。

怎样叫做阴阳呢?凡是一个人做善事被人知道,叫做阳善;做善事而别人不知道,叫做阴德。有阴德的人,上天自然会知道并且会报酬他的。有阳善的人,大家都晓得他,称赞他,他便享受世上的美名。享受好名声,虽然也是福,但是名这个东西,为天地所忌,天地是不喜欢爱名之人的。只要看世界上享受极大名声的人,而他实际上没有功德,可以称配他所享受的名声,常会遭遇到料想不到的横祸,一个人并没有过失差错,反倒被冤枉,无缘无故被人栽上恶名的人,他的子孙,常常会忽然间发达起来。这样看来,阴德和阳善的分别,真是细微得很,不可以不加以分辨啊!

怎样叫做是非呢?从前春秋时代的鲁国定有一种法律,凡是鲁国人被别的国家抓去做奴隶;若有人肯出钱,把这些人赎回来,就可以向官府领取赏金。但是孔子的学生子贡,他很有钱,虽然也替人赎回被抓去的人回来,子贡却是不肯接受鲁国的赏金。他不肯接受赏金,纯粹是帮助他人,本意是很好。但是孔子听到之后,很不高兴的说:这件事子贡做错了,凡是圣贤无论做什么事情,都是要做了以后,能把风俗变好;可以教训,引导百姓做好人,这种事才可以做;不是单单为了自己觉得爽快称心,就去做的。现在鲁国富有的人少,穷苦的人多;若是受了赏金就算是贪财;那末不肯受贪财之名的人,和钱不多的人,就不肯去赎人了。一定要很有钱的人,才会去赎人。如果这样的话,恐怕从此以后,就不会再有人向诸侯赎人了。

子路看见一个人,跌在水里,把他救了上来。那个人就送一只牛来答谢子路,子路就接受了。孔子知道了,很欣慰的说:从今以后,鲁国就会有很多人,自动到深水大河中去救人了。

由这两件事,用世俗的眼光来看,子贡不接受赏金是好的,子路接受牛,是不好的;不料孔子反而称赞子路,责备子贡。照这样看来,要知道一个人做善事,不能只看眼前的效果,而要讲究是不是会产生流传下去的弊端;不能只论一时的影响,而是要讲究长远的是非;不能只论个人的得失,而是要讲究它关系天下大众的影响。

现在所为,虽然是善,但是如果流传下去,对人有害,那就虽然像善,实在还不是善;现在所行,虽然不是善,但是如果流传下去,能够帮助人,那就虽然像不善,实在倒是善!这只不过是拿一件事情来讲讲罢了。说到其它种种,还有很多。例如:一个人应该做的事情,叫做义,但是有的时候,做该做的事,也会做错,做了倒反而坏事。

譬如坏人,可以不必宽放他,有人宽放他,这事情不能说不是义;但是宽放了这个坏人,反而使他的胆子更大,坏事做得更多;结果旁人受害,自己也犯罪;倒不如不要宽放他,给他儆戒,使他不再犯罪的好,不宽放他,是非义,使这个人不再犯罪,是义,这就叫做非义之义。

礼貌是人人应该有的,但是要有分寸,用礼貌对待人,是礼;但若是过份,反而使人骄傲起来,就成为非礼了,这就叫做非礼之礼。

信用虽要紧,但是也要看状况,譬如:顾全小的信用,是信;要顾全小信,却误了大事;反而使得大信,不能顾全,此变成非信了,这就叫做非信之信。

爱人本来是慈;但是因为过份的慈爱,反而使人胆子变大,闯出大祸,那就变成不慈了,这就叫做非慈之慈。这些问题,都应该细细地加以判断,分别清楚。

什么叫做偏正呢?从前明朝的宰相吕文懿公刚才辞掉宰相的官位,回到家乡来,因为他做官清廉,公正,全国的人都敬佩他,就像是群山拱卫著泰山,众星环绕著北斗星一样。独独有一个乡下人,喝醉酒后,骂吕公。但是吕公并没有因为被他骂而生气,并向自己的用人说:这个人喝酒醉了,不要和他计较。

吕公就关了门,不理睬他。过了一年,这个人犯了死罪入狱,吕公方才懊悔的讲:若是当时同他计较,将他送到官府治罪,可以藉小惩罚而收到大儆戒的效果,他就不至于犯下死罪了,我当时只想心存厚道,所以就轻轻放过他;那知道,反而养成他天不怕地不怕的亡命之徒的恶性。他以为就算是骂宰相,也没什么大不了,一直到犯下死罪,送了性命。这就是存善心,反倒做了恶事的一个例子。

也有存了恶心,倒反而做了善事的例子。像有一个大富人家,碰到荒年,穷人大白天在市场上抢米;这个大富人家,便告到县官那里;县官偏偏又不受理这个案子,穷人因此胆子更大,愈加放肆横行了。于是这个大富人家就私底下把抢米的人捉起来关,出他的丑,那些抢米的人,怕这大富人家捉人,反倒安定下来,不再抢了。若不是因为这样,市面上几乎大乱了。所以善是正,恶是偏,这是大家都知道的。但是也有存善心,反倒做了恶事的例子。

这是存心虽正,结果变成偏,只可称做正中的偏;不过也有存恶心,反倒做了善事的例子,这是存心虽是偏,结果反成正,只可称做偏中的正;这种道理大家不可不知道。怎样叫做半满的善呢?易经上说:一个人不积善,不会成就好的名誉;不积恶,则不会有杀身的大祸。

书经上说:商朝的罪孽,像穿的一串钱那么满;就仿佛收藏东西装满了一个容器里一样。

如果你很勤奋的,天天去储积,那么终有一天就会积满。商朝由开国一直到纣王,它的过失罪恶,到此时便积满了,因此迅速亡国。如果懒惰些,不去收藏积存,那就不会满。

所说的积善积恶,也像储存东西一样,这是讲半善满善的一种说法。

从前有一户人家的女子,到佛寺里去,想要送些钱给寺里,可惜身上没有多的钱,只有两文钱,就拿来布施给和尚。而寺里的首席和尚,竟然亲自替她在佛前回向,求忏悔灭罪。后来这位女子进了皇宫做了贵妃,富贵之后,便带了几千两的银子来寺里布施。但是这位主僧,却只是叫他的徒弟,替那个女子回向罢了,那个女子不懂前后两次的布施,为什么待遇差别如此之大?就问主僧说:我从前不过布施两文钱,师父就亲自替我忏悔。现在我布施了几千两银子,而师父不替我回向,不知是什么道理?

主僧回答她说:从前布施的银子虽然少,但是你布施的心,很真切虔诚,所以非我老和尚亲自替你忏悔,便不足以报答你布施的功德;现在布施的钱虽然多,但是你布施的心,不像从前真切,所以叫人代你忏悔,也就够了。这就是几千两银子的布施,只算是半善;而两文钱的布施,却算是满善,道理在此。

又汉朝人钟离把他炼丹的方法,传给吕洞宾,用丹点在铁上,就能变成黄金,可拿来救济世上的穷人。吕洞宾问钟离说:变了金,到底会不会再变回铁呢?

钟离回答说:五百年以后,仍旧要变回原来的铁。

吕洞宾又说:像这样就会害了五百年以后的人,我不愿意做这样的事情。

钟离教吕洞宾点铁成金,不过是试试他的心而已。现在知道吕洞宾存心善良,所以对他说:修仙要积满三千件功德,听你这句话,你的三千件功德,已经做圆满了。

这是半善满善的又一种讲法。一个人做善事,而内心不可叨念,仿佛自己做了一件不得了的善事;能够这样,那么就随便你所做的任何善事,都能够成功而且圆满。若是做了件善事,这个心就牢记在这件善事上;虽然一生都很勤勉的做善事,也只不过是半善而已。

譬如拿钱去救济人,要内不见布施的我,外不见受布施的人,中不见布施的钱,这才叫做三轮体空,也叫做一心清净。如果能够这样的布施,纵使布施不过一斗米,也可以种下无边无涯的福了;即使布施一文钱,也可以消除一千劫所造的罪了。如果这个心,不能够忘掉所做的善事;虽然用了二十万两黄金去救济别人,还是不能够得到圆满的福。这又是一种说法。

怎么叫做大善小善呢?从前有一个人,叫做卫仲达,在翰林院里做官,有一次被鬼卒把他的魂引到了阴间。阴间的主审判官,吩咐手下的书办,把他在阳间所做的善事、恶事两种册子送上来。等册子送到一看,他的恶事册子,多得竟摊满了一院子;而善事的册子,只不过像一支筷子那样小罢了。主审官又吩咐拿秤来秤秤看,那摊满院子的恶册子反而比较轻,而像一支筷子那样小卷的善册子反而比较重。卫仲达就问说:我年纪还不到四十岁,那会犯了这么多的过失罪恶呢?

主审官说:只要一个念头不正,就是罪恶,不必等到你去犯,譬如看见女色,动了坏念头,那就是犯过。

因此,卫仲达就问这善册子里记的是什么。主审官说:皇帝有一次曾想要兴建大工程,修三山地方的石桥。你上奏劝皇帝不要修,免得劳民伤财,这就是你的奏章底稿。

卫仲达说:我虽然讲过,但是皇帝不听,还是动工了,对那件事情的进行,并没有发生作用,这份疏表怎么还能有这样大的力量呢?

主审官说:皇帝虽然没有听你的建议,但是你这个念头,目的是要使千万百姓免去劳役;倘使皇帝听你的,那善的力量就更大了哩!

所以立志做善事,目的在利益天下国家百姓,那么善事纵然小,功德却很大。假使只为了利益自己一个人,那么善事虽然多,功德却很小。

怎么叫做难行易行的善呢?从前有学问的读书人,都说:克制自己的私欲,要从难除去的地方先除起。

孔子的弟子樊迟,问孔子怎样叫作仁?孔子也说,先要从难的地方下工夫。

孔子所说的难,也就是除掉私心;并应该先从最难做,最难克除的地方做起。一定要像江西的一位舒老先生,他在别人家教书,把两年所仅得的薪水,帮助一户穷人,还了他们所欠公家的钱,而免除他们夫妇被拆散的悲剧。

又像河北邯郸县的张老先生,看到一个穷人,把妻儿抵押了,钱也用了;若是没有钱去赎回,恐怕妻儿都要活不成了。于是就舍弃他十年的积蓄,替这个穷人赎回他的妻儿。像舒老先生,张老先生,都是在最难处,旁人不容易舍的,他们竟然能够舍得啊!

又像江苏省镇江的一位靳老先生,虽然年老没有儿子,他的穷邻居,愿意把一个年轻的女儿给他做妾,愿能为他生一个儿子。但是这位靳老先生不忍心误了她的青春,还是拒绝了,就把这女子送还邻居。这又是很难忍处,而能够忍得住的事呀!所以上天赐给他们这几位老先生的福,也特别的丰厚。

【\kai{善有真,也有假,有端有曲有阴阳,有是有非有偏正,有半有满有大小,有难有易当深辨呀,当深辨!}】

凡是有财有势的人要立些功德,比平常人来得容易,但是容易做,却不肯做,那就叫做自暴自弃了;而没钱没势的穷人,要做些福,都会有很大的因难,难做到而能做到,这才真是可贵啊!我们为人处事,应该遇到机缘,就去做救济众人的事。不过救济众人,也不是容易的事,救济众人的种类很多,简单的说,它的重要项目,大约有十种:

第一、是与人为善。看到别人有一点善心,我就帮他,使他善心增长。别人做善事,力量不够,做不成功,我就帮他,使他做成功,这都是与人为善。

第二、是爱敬存心。就是对比我学问好,年纪大,辈份高的人,都应该心存敬重。而对比我年纪小,辈份低,景况穷的人,都该要心存爱护。

第三、是成人之美。譬如一个人,要做件好事,尚未决定,则应该劝他尽心尽力去做。别人做善事时,遇到了阻碍;不能成功,应想方法,指引他,劝导他使得他成功;而不可生嫉妒心去破坏他。

第四、是劝人为善。碰到做恶的人,要劝他做恶绝对有苦报,恶事万万做不得。碰到不肯为善,或只肯做些小善的人,就要劝他行善绝对有好报,善事不但要做,而且还要做得多。做得大。

第五、是救人危急。一般人大多喜欢锦上添花,而缺乏雪中送炭的精神;而当遇到他人最危险、最困难、最紧急的关头;能及时向他伸出援手,拉他一把,出钱出力帮他解决危急困境,可以说是功德无量,但是不可以引以为傲!

第六、是兴建大利。有大利益的事情,自然要有大力量的人,才能做到,一个人既然有大力量,自然应该做些大利益的事情,以利益大众。例如,修筑水利系统、救济大灾害等等。但是没有大力量的人,也可以做到的。譬如,发现河堤上有个小洞,水从洞里冒出,只要用些泥土、小石,将小洞塞住,这堤防就可以保住,而防止了水灾的发生。事情虽然小,但这种功效也是不可忽视的。

第七、是舍财作福。俗语说:人为财死,世人的心总爱钱财,求财都来不及,还愿意去舍财济助他人吗?因此,能舍财去消除别人的灾难,解决他人的危急;对一个常人而言,已不简单,对穷人来说,则更加了不起。如按因果来讲,‘舍得,舍得,有舍才有得。’‘舍不得,舍不得,不舍就不得。’;做一分善事就会有一分福报,所以不必忧愁我们会因为舍财救人,而使自己的生活陷于绝路。

第八、是护持正法。这种法,就是指各种宗教的法。宗教有正,有邪,法也有正,有邪,邪教的邪法最害人心,自然应该禁止。而具有正知正见的佛法,是最容易劝导人心,挽回善良风俗的。若是有人破坏,一定要用全力保护维持,不可让他破坏。

第九、是敬重尊长。凡是学问深,见识好,职位高,辈份大,年纪老的人,都称为尊长。自己都应该敬重,不可看轻他们。

第十、是爱惜物命。凡是有性命的东西,虽然像蚂蚁那样小;也是有知觉的,晓得痛苦,并且也会贪生怕死。应该要哀怜它们,怎可以乱杀乱吃呢?有人常说:这些东西,本来就是要给人吃的。这话是最不通的,而且都是贪吃的人所造出来的话。

以上所讲的十种,只是大概的说明,下面是分别举例比喻:什么叫做与人善呢?从前虞朝的舜,在他还没有做君主之前,在雷泽湖边看见年轻力壮的渔夫,都拣湖水深处去抓鱼;而那些年老体弱的渔夫,都在水流得急而且水较浅的地方抓。

水流急,鱼停不住,浅滩水少,鱼也比较少,不比水深的地方,鱼都在那里游来游去,较容易抓。那些年轻力壮的渔夫,把好的地方都占去了。

舜看见这种情形,心里面悲伤哀怜他们。就想了一个方法,他自己也去参加捉鱼,看见那些喜欢抢夺的人,就把他们的过失,掩盖起来,而且也不对外讲;看见那些比较谦让的渔夫,便到处称赞他们,拿他们作榜样,并且学习他们谦让的模样。像这样,舜抓了一年的鱼,大家都把水深鱼多的地方让出来了。

舜的故事,不过是用来劝化人,不可误解是劝人抓鱼。要知道抓鱼是犯杀生的罪孽,千万不可以做啊!那么像舜那样明白聪明的圣人,那有不能说几句中肯的话,来教化众人,而一定要亲自参与呢?要晓得舜不用言语来教化众人,而是拿自己做榜样,使人见了,感觉渐愧而改变自己的自私心理,这真是一个用心良苦的人,所费的苦心啊!

我们生在这个人心风俗败坏,末世的时代,做人很不容易;因此,旁人有不如我的地方,不可以把自己的长处,去盖过旁人?旁人有不善的事情,不可以把自己的善,来和别人比较。别人能力不及我,不可以把自己有的能力,来为难别人。自己纵然有才干聪明,也要收敛起来,不可以外露炫耀,应该像是没有聪明才干一样。要看聪明才干,都是虚的、假的一般。

看到别人有过失,姑且替他包含掩盖。像这样,一方面可以使他有改过自新的机会,另一方面可以使他有所顾忌而不敢放肆。若是扯破面皮,他就没有顾忌了。

看到旁人有些小的长处,可以学的,或有小的善心善事,可以记的;都应该立刻翻转过来,放下自己的主见,学他的长处;并且称赞他,替他广为传扬。一个人在平常生活中,不论讲句话或是做件事,全不可为自己,发起一种自私自利的念头;而要全为了社会大众设想,立出一种规则来,使大众可以通行遵守,这才是一位伟大的人物,把天下所有的一切,都看做是公而不是私的度量呢!

什么叫做爱敬存心呢?君子与小人,从外貌来看,常常容易混淆,分不出真假。因为小人会装假仁假义,冒充君子。不过这一点存心,君子是善,小人是恶,彼此相去很远,他们的分别,就像黑白两种颜色,绝对相反不同。所以孟子说:君子所以与常人不同的地方,就是他们的存心啊!

君子所存的心,只有爱人敬人的心。因为人虽然有亲近的,疏远的,有尊贵的,有低微的,有聪明的,有愚笨的,有道德的,有下流的,千千万万不同的种类;但是这些都是我们的同胞,都是和我们一样有生命,有血有肉,有感情,那一个不该爱他敬他呢?爱敬众人,就是爱敬圣贤人。能够明白众人的意思,就是明白圣贤人的意思。为什么呢?

因为圣贤人本来都希望世界上的人,大家都能安居乐业,过著幸福美满的生活。所以,我们能够处处爱人,处处敬人,使世上的人,个个平安幸福,也就可以说是代替圣贤,使这个世界上人人都能够平安快乐了。

什么叫做成人之美呢?举例来说,若是把一块里面有玉的石头,随便乱丢抛弃,那末这块里面有玉的石头也只不过是和瓦片碎石一样,一文不值了。若是把它好好的加以雕刻琢磨,那么这块石头,就成了非常珍贵的宝物圭璋了。

一个人也是如此,也全是靠劝导提引;所以看到别人做一件善事,或者是这个人立志向上,而且他的资质足以造就的话;都应该好好的引导他,提拔他,使他成为社会上的有用之材;或是夸赞他,激励他,扶持他;若是有人冤枉他,就替他辩解冤屈,来替他分担无端被人恶意的毁谤,可以设法代替他,顶替他被毁谤的事实,减轻他所受的毁谤,这样叫做分谤。

务必要使他能够立身于社会,而后才算是尽了我的心意。大概通常的人,对那些与他不同类型的人,都不免有厌恶感,譬如小人恨君子,恶人恨善人。

在同一个乡里的人,都是善的少,不善的多。正因为不善的人很多,善的人少,所以善人处在世俗里,常常被恶人欺负,很难立得住脚,况且豪杰的性情大多数是刚正不屈,并且不注意修饰外表,世俗的眼光,见识不高,只看外表,就说长道短,随便批评;所以做善事也常常容易失败,善人也常常被人毁谤。

碰到这种情形,只有全靠仁人长者,才能纠正那些邪恶不正的人,教导指引他们改邪归正,保护,帮助善人,使他成立;像这样辟邪显正的功德,实在是最大的。

什么叫做劝人为善呢?一个人既然已经生在世上做了人,那一个没有良心呢?但是因为汲汲地追逐名利,弄得这世间忙碌不堪,只要有名利可得,就昧著良心,不择手段地去做,那就最容易堕落了。所以与别人往来相处,时常要留心观察这个人,若是看他要堕落了,就应该随时随地提醒他,警告他,开发他的糊涂昏乱。

譬如,看见他在长夜里做了一个浑浑噩噩的梦,一定要叫唤他,使他赶快清醒;又譬如看他长久陷落在烦恼里,一定要提拔他一把,使他头脑转为清凉。

像这样以恩待人,功德是最周遍,最广大的了。从前韩文公曾说:以口来劝人,只在一时,事情过了,也就忘了;并且别处的人,无法听到。以书来劝人,可以流传到百世,并且能传遍世界;所以做善书,有立言的大功德。

这里说以口来劝,用书来劝人为善,与前面所讲的与人为善比较起来,虽然较注重形式的痕迹,但是这种对症下药的事,时常会有特殊的效果;这种方法是不可以放弃的。

并且劝人也得要劝的得当,譬如这个人太倔强,不可以用话来劝,你若是用话去劝了,不但是白劝,所劝的话,也成了废话,这叫做失言。如果这个人性情温顺,可以用话来劝,你却是不劝,错过了劝人为善的机会,这叫做失人。失言失人,都是自己智慧不够,分辨不出来,就应该自己仔细反省检讨;如此才能不失言,也不失人。

什么叫做救人危急呢?患难颠沛的事情,在人的一生当中,都是常有的。假使偶而碰到患难危急的人,应该要将他的痛苦,当做是发生在自己的身上一样,赶快设法解救,看他有什么被人冤屈压迫的事情,或是用话语帮助他申辩明白,或是用种种的方法来救济他的困苦。明朝的崔子曾经说:恩惠不在乎大小,只要在别人危急的时候,赶紧去帮助他就可以了。这句话真正是仁者的话呀!

什么叫做兴建大利呢?讲小的,在一个乡中,讲大的,在一个县内,凡是有益公众的事,最应该发起兴建。或是开辟水道来灌溉农田;或是建筑堤岸来预防水灾;或是修筑桥梁,使行旅交通方便,或是施送茶饭,救济饥饿口渴的人。

随时遇到机会,都要劝导大家,同心协力,出钱出力来兴建;纵然有别人在暗中毁谤你,中伤你;你也不要为了避嫌疑就不去做,也不要怕辛苦,担心别人嫉妒怨恨,就推托不做,这都是不可以的。

什么叫做舍财作福呢?佛门里的万种善行,以布施为最重要。讲到布施,就只有一个舍字,什么都舍得,就合佛的意思了。

真正明白道理的人,什么都肯舍;譬如自己身上的眼睛,耳朵,鼻子,舌头,身体,念头,没有一样不肯舍掉。譬如,佛陀曾在因地修行的时候,舍身饲虎。

在身外的色、声、香、味、触、法,也都可以一概舍弃。一个人所有的一切,没有一样不可以舍掉,能够如此,那就身心清净,没有烦恼,就如同佛菩萨了。

若是不能什么都舍,那就先从钱财上著手布施。世间人都把穿衣吃饭,看得像生命一样重要;因此,钱财上的布施也最为重要;如果我能够痛痛快快地施舍钱财;对内而言,可以破除我小器的毛病;对外而言,则可救济别人的急难。

不过钱财不易看破,起初做起来,难免会有一些勉强,只要舍惯了,心中自然安逸,也就没有什么舍不得了。这是最容易消除自己的贪念私心,也可以除掉自己对钱财的执著与吝啬。

什么叫做护持正法呢?法是千万年来,有灵性的有情生命的眼目,也是真理的准绳;但是法有正有邪,如果没有正法,如何能够参加帮助天地造化之功呢?怎样会使得各式各样的人以及种种的东西,都能够像裁布成衣那样的成功呢?怎样可以脱出那种种的迷惑,离开那种种的束缚呢?怎样可以建设整理世上一切的事情,和逃出这个污秽世界,生死轮回的苦海呢?这都需要靠有了正法,才像有了光明的大路可走。

所以凡是看到圣贤的寺庙,图像,经典,遗训,都要加以敬重;至于有破损不完全的,都应该要修补,整理。而讲到佛门正法,尤其应该敬重的加以传播宣扬,使大家都重视,才可以上报佛的恩德,这些都是更应该加以全力去实践的。

什么叫做敬重尊长呢?家里的父亲,兄长,国家的君王,长官;以及凡是年岁,道德,职位,见识高的人,都应该格外虔诚的去敬重他们。

在家里侍奉父母,要有深爱父母的心,与委婉和顺的容貌;而且声要和,气要平;这样不断地薰染成习惯,就变成自然的好性情,这就是和气可以感勤天心的根本办法。

出门在外侍候君王,不论什么事,都应该依照国法去做;不可以为君王不知道,自己就可以随意乱做呀!办一个犯罪的人,不论他的罪轻或重,都要仔细审问,公平地执法;不可以为君王不知道,就可以作威作福冤枉人!

服侍君王,像面对上天一样的恭敬,这是古人所订的规范,这种地方关系阴德最大。你们试看,凡是忠孝人家,他们的子孙,没有不发达久远而且前途兴旺的,所以一定要小心谨慎的去做。

什么叫做爱惜物命呢?要知道一个人之所以能够算他是人,就是在他有这一片恻隐的心罢了。所以孟子说:没有恻隐之心就不是人。

求仁的,就是求这一片恻隐之心;积德的,也就是积这一片恻隐的心。有恻隐心就是仁;有恻隐心,就是德。没有恻隐心,就是无仁心,没道德。周礼上曾说:每年正月的时候,正是畜牲最容易怀孕的期间,这时候祭品勿用母的。因为要预防畜牲肚里有胎儿的缘故。

孟子说:君子不肯住在厨房附近。就是要保全自己的恻隐之心,所以,前辈有四种肉不吃的禁忌。譬如说,听到动物被杀的声音,不吃,或者在它被杀的时候看见,不吃;或者是自己养大的,不吃,或专门为我杀的,不吃。后辈的人,若要学习前辈的仁慈心,一下子做不到断食荤腥,也应该依照前辈的办法,禁戒少吃。

照佛法来讲,一切有生命的东西,都是因为前生造了孽而投胎做畜牲;等到它们的罪孽还完了后,仍然可以投胎做人的。做人以后若是肯修行,也可以修成佛。那么我今生所吃的肉,难保不就是吃了未来佛的肉,并且现在的畜牲,在无量过去的前世中,也一定曾经做过人;那么它们可能曾做过我前生中的父母,妻子,亲族,朋友,我今天吃的肉,可能就是吃我前生的父母,妻子,亲族,朋友的肉了。而今天我做人,它做畜牲,我吃它,我就造了杀孽,与它结下冤仇。如果被我吃的畜牲,来世它的孽债还清了,投生做了人,而我却因为杀生造孽,投胎做畜牲,恐怕他也要报复我杀他之仇,而来杀我,吃我了。这样说来,还能杀生么?肉还能吃得下吗?况且吃肉就算味道好,也不过是经过嘴里到喉咙那段时候,还觉得有味道,等到咽了下去,还有什么味道?与素菜有什么两样,为什么一定要杀生造孽呢?

虽然一时做不到不吃肉,也应该渐渐地减少吃肉,直到完全不吃。这样子慈悲心就会愈来愈增加。不但杀生应戒,就是那些极小,不论愚蠢的或是有灵性的,凡是有生命的,都应该禁止伤害它们的性命。像要用丝来做衣服,就把蚕茧放在水里烧,那要伤害多少蚕的性命?掘地种田,要杀害地下多少虫的性命;想想我们穿的衣、吃的饭,是从那里来的呢?都是杀它们的命,来养活我们自己;所以糟蹋粮食,浪费东西的罪孽,实在也应该与杀生的罪孽相等。至于随手误伤的生命,脚下误踏而死的生命,又不晓得有多少,这都应该要设法防止。宋朝的苏东坡有首诗说:爱鼠常留饭,怜蛾不点灯。

意思是说:恐怕老鼠饿死,所以为老鼠留些饭;哀怜飞蛾扑到灯上烫死,所以灯也不点。这话是多么的仁厚慈悲呀!

善事无穷无尽,那能说得完;只要把上边说的十件事,加以推广发扬,那么无数的功德,就都完备了。

【\kai{救济众人事,种类有很多,简单而言之,大概有十种,与人为善一呀!爱敬存心二,成人之美三呀!劝人为善四,救人危急五呀!兴建大利六,舍财作福七呀!护持正法八,敬重尊长九呀!爱惜物命十呀,爱惜物命十。}】

\chapter{第四篇\ 谦德之效}
第三篇所说的,都是积善的方法,能够积善,自然最好,但人在社会上,不能不和人来往,做人的方法必须加以讲究;而最好的方法就是谦虚了。一个人能谦虚,在社会上一定会得到大众广泛的支持与信任,而懂得谦虚,便更知道‘日新又新’的重要;不但学问要求进步,做人做事交朋友等等,样样都要求进步。所有种种的好处,都从谦虚上得来,所以称为谦德。这一篇专讲谦虚的好处,谦虚的效验;大家要仔细的研究,不可以囫囵吞枣,那就必定能够得到大的利益。

【\kai{书经说:满招损呀!谦受益,自满的就要招损害;谦虚的就会受到益呀!受到益。}】

易经谦卦上说:天的道理,不论什么,凡是骄傲自满的,就要使他亏损,而谦虚的就让他得到益处。地的道理,不论什么,凡是骄傲自满的,也要使他改变,不能让他永远满足;而谦虚的要使他滋润不枯,就像低的地方,流水经过,必定会充满了他的缺陷。鬼神的道理,凡是骄傲自满的,就要使他受害,谦虚的便使他受福。人的道理,都是厌恶骄傲自满的人,而喜欢谦虚的人。

这样看来,天、地、鬼、神、人、都看重谦虚的一边。易经上六十四卦,所讲的都是天地阴阳变化的道理,教人做人的方法。每一卦爻中,有凶有吉,凶卦是警戒人去恶从善,吉卦是勉励人要日新又新,唯有这个谦卦,每一爻都吉祥。书经上也讲:自满,就会遭到损害,自谦,就会受到益处。

我好几次和许多人去参加考试,每次都看到贫寒的读书人,快要发达考中的时候,脸上一定有一片谦和,而且安详的光采发出来,仿佛可以用手捧住的样子。

辛未年,我到京城去会试,我的同乡嘉善人一起去参加会试的,大约有十个人,只有丁敬宇,这个人最年轻,而且非常谦虚,我告诉同去会试的费锦坡讲:这位老兄,今年一定考中。费锦坡问我说:怎样能看出来呢?

我说:只有谦虚的人,可以承受福报。老兄你看我们十人当中,有诚实厚道,一切事情,不敢抢在人前,像敬宇的吗?有恭恭敬敬,一切多肯顺受,小心谦逊,像敬宇的吗?有受人侮辱而不回答,听到人家毁谤他而不去争辩,像敬宇的吗?一个人能够做到这样,就是天地鬼神,也都要保佑他,岂有不发达的道理?等到放榜,丁敬宇果然考中了。

丁丑年在京城里,和冯开之住在一起,看见他总是虚心自谦,面容和顺,一点也不骄傲,大大的改变了他小时候的那种习气。他有一位正直又诚实的朋友季霁岩,时常当面指责他的错处,但却只看到他,平心静气地接受朋友的责备,从来不反驳一句话。我告诉他说:一个人有福,一定有福的根苗;有祸,也一定有祸的预兆。只要这个心能够谦虚,上天一定会帮助他,你老兄今年必定能够登第了!后来冯开之果然真的考中了。

赵裕峰,名光远,是山东省冠县人;不满二十岁的时候,就中了举人,后来又考会试,却多次不中。他的父亲做嘉善县的主任秘书,裕峰随同他父亲上任。裕峰非常羡慕嘉善县名士钱明吾的学问,就拿自己的文章去见他,那晓得这位钱先生,竟然拿起笔来,把他的文章都涂掉了。裕峰不但不发火,并且心服口服,赶紧把自己文章的缺失改了。如此虚心用功的年轻人,实在是少有,到了明年,裕峰就考中了。

壬辰年我入京城去觐见皇帝,见到一位叫夏建所的读书人,看到他的气质,虚怀若谷,毫无一点骄傲的神气,而且他那谦虚的光采,就像会逼近人的样子。我回来告诉朋友说:凡是上天要使这个人发达,在没有发他的福时,一定先发他的智慧,这种智慧一发,那就使浮滑的人自然会变得诚实,放肆的人也就自动收敛了,建所他温和善良到这种地步,是已发了智慧了,上天一定要发他的福了。等到放榜的时候,建所果然考中了。

江阴有一位读书人。名叫张畏岩,他的学问积得很深,文章做得很好,在许多读书人当中,很有名声。甲午年南京乡试,他借住在一处去寺院里,等到放榜,榜上没有他的名字,他不服气,大骂考官,眼睛不清楚,看不出他的文章好。那时候有一个道士在旁微笑,张畏岩马上就把怒火发在道士的身上。道士说:你的文章一定不好。张畏岩更加的发怒说:你没有看到我的文章,怎么知道我写得不好呢?道士说:我常听人说,做文章最要紧的,是心平气和,现在听到你大骂考官,表示你的心非常不平,气也太暴了,你的文章怎么会好呢?

张畏岩听了道士的话,倒不觉的屈服了,因此,就转过来向道士请教。道士说:要考中功名,全要靠命,命里不该中,文章虽好,也没益处,仍不会考中,一定要你自己改变改变。

张畏岩问道:既然是命,怎样去改变呢?道士说:造命的权,虽然在天,立命的权,还是在我;只要你肯尽力去做善事,多积阴德,什么福不可求得呢?

张畏岩说:我是一个穷读书人,能做什么善事呢?

道士说:行善事,积阴功,都是从这个心做出来的。只要常常存做善事,积阴功的心,功德就无量无边了。就像谦虚这件事,又不要花钱,你为什么不自我反省,自己工夫太浅,不能谦虚,反而骂考官不公平呢?

张畏岩听了道士的话,从此以后就压低一向骄傲的志向,自己很留意把持住自己,勿走错了路,天天加功夫去修善,天天加功夫去积德。到了丁酉年,有一天,他做梦到一处很高的房屋里去,看到一本考试录取的名册,中间有许多的缺行。他看不懂,就问旁边的人说:这是什么?那个人说:这是今年考试录取的名册。而张畏岩问:为什么名册内有这么多的缺行?那个人又回答说:阴间对那些考试的人,每三年考查一次,一定要积德,没有过失,这册里才会有名字。像名册前面的缺额,都是从前本该考中,但是因为他们最近犯了有罪过的事情,才把名字去掉的。

后来那个人又指了一行说:你三年来,很留心的把持住自己,没犯罪过,或者是应该补上这个空缺了,希望你珍重自爱,勿犯过失!果然张畏岩就在这次的会考,考中了第一百零五名。

【\kai{造命的权在天,立命的权在我,只要肯努力,多做善事积阴德呀!积阴德,什么福报求不得呀?求不得?}】

从上面所讲的看来,举头三尺高,一定有神明在监察著人的行为。因此,利人,吉祥的事情,都应该赶快的去做;凶险,损人的事,应该避免,不要去做,这是可以由我自己决定的,只要我存好心,约束一切不善的行为,丝毫不得罪天地鬼神,而且还要虚心,自己肯迁就不骄傲,使得天地鬼神,时时哀怜我,才可以有福的根基,那些满怀傲气的人,一定不是远大的器量,就算能发达,也不会长久地享受福报。稍有见识的人,一定不肯把自己肚量,弄得很狭窄,而自己拒绝可以得到的福,况且谦虚的人,他还有地方可以受到教导,若人不谦虚,谁肯去教他?

并且谦虚的人,肯学别人的好处,别人有善的行动,就去学他,那么得到的善行,就没有穷尽了。尤其是进德修业的人,一定所不可缺少的啊!

【\kai{举头三尺高呀!决定有神明,不但要存好心,而且要虚心,不可以做坏事,还要肯迁就,天地鬼神呀,千万莫得罪啊!莫得罪!}】

古人有几句老话说:有心要求功名的,一定可以得到功名;有心要求富贵的,一定可以得到富贵。一个人有远大的志向,就像树有根一样;树有根,就会生出丫枝花叶来。

人要立定了这种伟大的志向,必须在每一个念头上,都要谦虚,即使碰到像灰尘一样极小的事情,也要使别人方便,能够做到这样,自然会感动天地了。

而造福全在我自己,自己真心要造,就能够造成。像现在那些求取功名的人,当初那有什么真心,不过是一时的兴致罢了;兴致来了,就去求,兴致退了,就停止,孟子对齐宣王说:大王喜好音乐,若是到了极点,那么齐国的国运大概可以兴旺了。但是大王喜好音乐,只是个人在追求快乐罢了,若是能把个人追求快乐的心,推广到与民同乐,使百姓都快乐,那么齐国还有不兴旺的么?

我看求科名,也是这样,要把求科名的心,落实推广到积德行善上;并且要尽心尽力地去做,那么命运与福报,就都能够由我自己决定了!

各位听众朋友,在听完了《了凡四训有声书》之后,您的内心一定是感触很多,觉得获益匪浅,这的的确确是一部扣人心弦,净化人心的有声书;而了凡四训这本书的原文,文字非常的优美典雅,表面上看起来,好像并不难了解,但是其中所深藏宇宙人生的道理,却是非常的深奥,必须用心体会,才能有得于心。

所以盼望各位听众朋友,在听过录音带以后,最好能够熟读了凡四训的原文三百遍;必定能够信心益增,效法了凡先生立命精神的决心,会更加的坚定;进而身体力行,断恶修善。于是而个人的学业,事业,家庭均能圆满,成圣成贤,也是指日可待的。

为了利益大众,广为流传,在此我们声明,了凡四训有声书并没有所谓‘智慧财产权’的问题,非常欢迎善心人士大力提倡翻录,提供各级学校教学参考,来协助我们大、中、小学生的心理建设;并且能净化社会人心,提升道德观念;大家都学了凡精神,诸恶莫作,众善奉行,行善积德,服务人群,如果说人人能够如此,国家前途一定光明,世界一片祥和。

\part{附录}
\chapter{袁了凡居士传}
袁了凡先生,本名袁黄,字坤仪,江苏省吴江县人。年轻时入赘到浙江省嘉善县姓\xpinyin*{殳}的人家;因此,在嘉善县得了公费做县里的公读生。他于明穆宗隆庆四年(公元1570年),在乡里中了举人;明神宗万历十四年(公元1586年)考上进士,奉命到河北省宝坻县做县长。过了七年升拔为兵部「职方司」的主管人,任中碰到日寇侵犯朝鲜,朝鲜向中国求救兵。当时的「经略」(驻朝鲜军事长官)宋应昌奏准请了凡为「军前赞画」(参谋长)的职务,并兼督导支援朝鲜的军队。提督李如松掌握兵权,假装赐给高官俸禄与日寇谈和,日寇信以为真,没有设防;李如松发动突击,攻破形势险要的平壤,因而打败了日寇。

了凡先生因为这件事当面指责李如松,不应用诡诈的手段对付日寇,这样有损大明朝的国威;而且李如松手下的士兵随便杀害百姓,并以头来记功。了凡向李如松据理力争,李如松发怒;不但不接受劝诫,反而独自带着军队东走,使得了凡所率领的军队孤立无援。日寇因而乘机攻击了凡的军队,幸赖了凡机智应对,将日寇击退。而李如松的军队,最后终于被日寇击败了;他想要脱却自己的罪状,反而以十项罪名弹劾袁了凡;了凡很快地被提出审判,终于在「拾遗」(谏官)的仕内,被迫停职返乡。在家里,了凡非常恳切,认真地行善直到去世,过世时享年七十四岁。

明熹宗天启年间,了凡的冤案终于真相大白,朝廷追叙了凡征讨日寇的功绩,赠封他为「尚宝司少卿」的官衔。

了凡先生从当学生时,就非常喜欢研究学问,书不论古今,事不分轻重,他都认真研究,并且非常通达。例如:星象,法律,水利,理数,兵备,政治,堪舆等。

了凡先生在宝坻县当县长时,非常注重人民的福利,常常想做些有利地方的事情;宝坻县当时常有水灾泛滥,了凡先生于是积极兴办水利,将三汊河疏通,筑堤防以抵挡水患侵袭;并且教导百姓沿着海岸种植柳树,每当海水泛滥,挟带沙土冲上岸时,遇到柳树就积挡下来,久而久之变成一道堤防。于是了凡先生又督导百姓在堤防上建造沟渠,并鼓励百姓耕种;因此,荒废的土地渐渐地开垦,了凡先生又免除百姓种种杂役以便民,使得百姓安居乐业。

了凡先生家里并不富有,可是却非常喜欢布施,家居生活俭朴,每天诵经持咒,参禅打坐,修习止观。不管公私事务再忙,早晚定课从不间断。在这当中,了凡先生写下四篇短文,当时命名为「戒子文」,用来训诫他儿子,就是后来广行于世的「了凡四训」这本书。

了凡先生的夫人非常贤慧,经常帮助他行善布施,并且依照功过格记下所做的功德,因为她没有读过书,不会写字;因此用鹅毛管沾红墨水,每天在历书上做记号。有时了凡先生较忙,当天所做功德较少,她就皱眉头,希望先生能多做些善事。有一次,她为儿子裁制冬天的大袍子,想买棉絮做内里。

了凡先生问:“家里有丝绵又轻又暖和,为什么还买棉絮呢?”

了凡夫人答:“丝绵较贵,棉絮便宜,我想将家里的丝绵拿去换棉絮,这样可以多裁几件棉袄,赠送给贫寒的人家过冬!”

了凡先生听了非常高兴说:“你这样虔诚的布施,不怕我们孩子没有福报了!”

他们的儿子袁\xpinyin*{俨},后来中了进士,最后以广东省高要县的县长退休。

\chapter{云谷先大师传}
师讳法会,别号云谷,嘉善\xpinyin*{胥}山怀氏子。生于弘治庚申,幼志出世,投\xpinyin*{邑}大云寺某公为师。初习瑜伽(1),师每思曰:“出家以生死大事为切,何以碌碌衣食计为?”年十九,即决志操方(2),寻登坛受具。闻天台小止观法门,专精修习。法舟济禅师(3),续径山之道(4),掩关于郡之天宁。师往参扣,呈其所修。舟曰:“止观之要,不依身心气息,内外脱然。子之所修,流于下乘,岂西来的意耶?学道必以悟心为主。”师悲仰请益,舟授以念佛审实话头(5),直令重下疑情。师依教日夜参究,寝食俱废。一日受食,食尽亦不自知,碗忽堕地,猛然有省,恍如梦觉。复请益舟,乃蒙印可。阅《宗镜录》,大悟唯心之旨。从此一切经教,及诸祖公案,了然如睹家中故物。于是韬晦丛林,陆沉贱役。一日阅《\xpinyin*{镡}津集》,见明教大师(6)护法深心,初礼观音大士,日夜称名十万声。师愿效其行,遂顶戴观音大士像,通宵不\xpinyin*{寐},礼拜经行,终身不懈。

时江南佛法禅道,绝然无闻。师初至金陵,寓天界\xpinyin*{毘}卢阁下行道,见者称异。魏国先王闻之,乃请于西园丛桂庵供养,师住此入定三日夜。居无何(7),予先太师祖西林翁(8),掌僧录,兼报恩住持,往谒师,即请住本寺之三藏殿。师危坐一\xpinyin*{龛},绝无将迎,足不越\xpinyin*{阃}(9)者三年,人无知者。偶有权贵人游至,见师端坐,以为无礼,谩辱之。师\xpinyin*曳{杖}之摄山栖霞(10)。

栖霞乃梁朝开山,武帝凿千佛岭,累朝赐供\xpinyin*{赡}田地。道场荒废,殿堂为虎狼巢。师爱其幽深,遂诛茅(11)于千佛岭下,影不出山。时有盗侵师,窃去所有,夜行至天明,尚不离庵。人获之,送至师。师食以饮食,尽与所有持去,由是闻者感化。太宰五台陆公,初仕为祠部主政,访古道场,偶游栖霞,见师气宇不凡,雅重之。信宿(12)山中,欲重兴其寺,请师为住持。师坚辞,举嵩山善公以应命。善公尽复寺故业,斥豪民占据第宅,为方丈、建禅堂、开讲席、纳四来。江南丛林\xpinyin*{肇}于此,师之力也。

道场既开,往来者众,师乃移居于山之最深处,曰“天开岩”,吊影如初。一时宰官居士,因陆公开导,多知有禅道,闻师之风,往往造谒。凡参请者,一见,师即问曰:“日用事如何?”无论贵贱僧俗,入室必掷蒲团于地,令其端坐,返观自己本来面目,甚至终日竟夜无一语。临别必叮咛曰:“无空过日。”再见,必问别后用心功夫,难易若何。故荒唐者,茫无以应。以慈愈切而严益重,虽无门庭设施,见者望崖不寒而栗。然师一以等心相摄,从来接人软语低声,一味平怀,未尝有辞色(13)。士大夫归依者日益众,即不能入山,有请见者,师以化导为心,亦就见(14)。岁一往来城中,必主于回光寺。每至,则在家二众,归之如绕华座。师一视如幻化人,曾无一念分别心。故亲近者,如婴儿之傍慈母也。出城多主于普德,\xpinyin*{臞鹤}悦公实禀其教。

先太师翁,每延入丈室,动经旬月。予童子时,即亲近执侍,辱师器之,训诲不倦。予年十九,有不欲出家意。师知之,问曰:“汝何背初心耶?”予曰:“第厌其俗耳。”师曰:“汝知厌俗,何不学高僧?古之高僧,天子不以臣礼待之,父母不以子礼\xpinyin*{畜}之。天龙恭敬,不以为喜。当取《传灯录》、《高僧传》读之,则知之矣。”予即简书笥,得《中峰广录》一部,持白师。师曰:“熟味此,即知僧之为贵也。”予由是决志\xpinyin*{薙}染(15),实蒙师之开发,乃嘉靖甲子岁也。丙寅冬,师\xpinyin*{愍}禅道绝响,乃集五十三人,结坐禅期于天界。师力拔予入众同参,指示向上一路,教以念佛审实话头,是时始知有宗门事(16)。比南都诸\xpinyin{刹}{cha4}(17),从禅道者四五人耳。

师垂老,悲心益切。虽最小沙弥,一以慈眼视之,遇之以礼,凡动静威仪,无不耳提面命,循循善诱,见者人人以为亲己。然护法心深,不轻初学,不慢毁戒。诸山僧多不律,凡有干法纪者,师一闻之,不待求而往救,必恳恳当事(18),佛法付嘱王臣为外护,惟在仰体佛心,辱僧即辱佛也。闻者莫不改容释然,必至解脱而后已,然竟\xpinyin*{罔}闻于人者。故听者,亦未尝以多事为烦。久久,皆知出于无缘慈也。了凡袁公未第时,参师于山中,相对默坐三日夜,师示之以唯心立命之旨。公奉教事,详《省身录》。由是师道日益重。隆庆辛未,予辞师北游。师诫之曰:“古人行脚,单为求明己躬下事,尔当思他日将何以见父母师友,慎毋虚费草鞋钱也。”予涕泣礼别。

壬申春,嘉禾吏部尚书默泉吴公、刑部尚书旦泉郑公、平湖太仆五台陆公与弟云台,同请师故山(19)。诸公时时入室问道,每见必炷香请益,执弟子礼。达观可禅师,常同尚书平泉陆公、中书思庵徐公,谒师扣《华严》宗旨。师为发挥四法界圆融之妙,皆叹未曾有。

师寻常示人,特揭唯心净土法门,生平任缘,未常树立门庭。诸山但有禅讲道场,必请坐方丈。至则举扬百丈规矩,务明先德典刑(20),不少假借。居恒安重寡言,出语如空谷音。定力摄持,住山清修,四十余年如一日,胁不至席。终身礼诵,未尝辍一夕。当江南禅道草\xpinyin*{昧}(21)之时,出入多口之地,始终无议之者,其操行可知已。

师居乡三载,所蒙化千万计。一夜,四乡之人,见师庵中大火发。及明趋视,师已寂然而逝矣,万历三年乙亥正月初五日也。师生于弘治庚申,世寿七十有五,僧腊五十。弟子真印等,\xpinyin*{荼毗}葬于寺右。

予自离师,遍历诸方,所参知识,未见操履平实、真慈安详之若师者。每一兴想,师之音声色相,昭然心目。以感法乳之深,故至老而不能忘也。师之发迹入道因缘,盖常亲蒙开示。第末后一着,未知所归。前丁巳岁,东游,赴沈定凡居士斋。礼师塔于栖真,乃募建塔亭,置供赡田,少尽一念。见了凡先生铭未悉,乃概述见闻行履为之传,以示来者。师为中兴禅道之祖,惜机语失录,无以发扬秘妙耳。

释德清曰:达摩单传之道,五宗而下,至我明径山之后,狮弦(22)将绝响矣。唯我大师,从法舟禅师,续如线之脉。虽未大建法\xpinyin*{幢},然当大法草昧之时,挺然力振其道,使人知有向上事。其于见地稳密,操履平实,动静不忘规矩,犹存百丈之典刑。遍阅诸方,纵有作者(23),无以越之。岂非一代人天师表欤!清愧钝根下劣,不能克绍家声,有负明教。至若荷法之心,未敢忘于一息也。敬述师生平之概,后之观者,当有以见古人云。
\end{document}
